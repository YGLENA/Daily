\mytitle{Field-theory approach to quantum many-body systems}
Yesterday, we have considered the system with filling factor $\nu=p'/q'$, where $p',q'$ are coprimes. In this case, the Lagrangian allows only the $\cos(2p\phi)$ terms when $p=nq'$ for some $n\in \mathbb{Z}$. Now suppose that $cos(2q'\phi)$ is allowed and $q'^2\kappa<2$. Then the term is relevant, and $\phi$ is pinned to the potential minimum, which shows the gap of the system. Furthermore, we can see that when the 1-dimensional system with $\nu=p'/q'$ gains a gap, then there are $q'$ degenerate ground states, which are related by $\phi\mapsto \phi+n\frac{\pi}{q'}$ for $n=0,1,\cdots,q'-1$. This is related to the spontaneous symmetry breaking of the lattice translational symmetry.

Indeed, if we consider $\nu=p'/q'$ with $q'>1$, then the particles tend to move more freely, which implies the gapless energy spectrum. To open a gap, the particles must be "locked" on the lattice, so that they cannot move freely. Indeed, consider $\nu=1/3$ case. Then we can lock the particles 
\noindent\rule{\textwidth}{1pt}
\newline