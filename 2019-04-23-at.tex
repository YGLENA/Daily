\mytitle{Algebraic Topology}
\begin{lemma} Let $A$ be the deformation retract of $X$. If $A$ is path connected, then $X$ is path connected.\marginnote{This can be more easily proven if we use the homology, or especially, $H_0$.}
\end{lemma}
\begin{proof}
Take the deformation retract $F:X\times I\rightarrow X$. For any two points $x_0,x_1\in X$, define $f_0(s)=F(x_0,s)$ and $f_1(s)=F(x_1,s)$. Then $f_0$ is a path from $x_0$ and $a\in A$, and $f_1$ is a path from $x_1$ and $b\in A$. Finally, since $A$ is path connected, there is a path $g$ connecting $a,b$. Then the path $(f_0\cdot g)\cdot \bar{f}_1$ is a path connecting $x_0$ and $x_1$.
\end{proof}
\begin{cor}[Wedge sum.] Let the for each space $X_\alpha$ there is a basepoint $x_\alpha$ and its open neighborhood $U_\alpha$ which deformation retracts to $x_\alpha$. Then the wedge sum $\vee_\alpha X_\alpha$ identifying their basepoints has the fundamental group $\pi_1(\vee_\alpha X_\alpha)\simeq *_\alpha \pi_1(X_\alpha,x_\alpha)$.
\end{cor}
\begin{proof} Take $A_\alpha=X_\alpha\vee\left(\vee_{\beta\neq \alpha}U_\beta\right)$. Then the intersection of two or more distinct $A_\alpha$ is $\vee_\alpha U_\alpha$. Since $U_\alpha$ has a deformation retract $F_\alpha$ to $x_0$, $\vee_\alpha U_\alpha$ has a deformation retract to $x_0$, which is defined from $F_\alpha$, which is well-defined since $F_\alpha|_{x_0}=1_{x_0}$ and continuous by pasting lemma. Since one-point set is path connected, $\vee_\alpha U_\alpha$ is path connected, thus we can use the van Kampen's theorem. Since $\vee_\alpha U_\alpha$ is simply connected, $\pi_1(X_\alpha)\simeq \pi_1(A_\alpha)$ and $i_{\alpha\beta}$ is a trivial map sending trivial loop to trivial loop, hence $N$ is a trivial group. Therefore $*_\alpha\pi_1(X_\alpha)\simeq \pi_1(X)$.
\end{proof}

\begin{exmp}[Loop deleted from $\mathbb{R}^3$.] Consider $X=\mathbb{R}^3-S^1$ where $S^1$ lies on the $xy$ plane. We can split this into right and left side with a little intersection, $A_R$ and $A_L$, then since $A_{R,L}$ can deformation retract to $S^1$, $\pi_1(A_{R,L})\simeq \mathbb{Z}$, and since $A_R\cap A_L$ can deformation retract to $S^1\vee S^1$ we get, due to the van Kampen's theorem or due to the corollary above, $\pi_1(A_{R}\cap A_L)\simeq \mathbb{Z}*\mathbb{Z}$. Now write the generator of $\pi_1(A_{R,L})$ as $r,l$ respectively, and the generator of $\pi_1(A_{R}\cap A_L)$ as $a$ and $b$. Then the inclusion map $i_{RL}:\pi_1(A_R\cap A_L)\rightarrow \pi_1(A_R)$ is the map with $i_{RL}(a)=i_{RL}(b)=r$, and samely $i_{LR}:\pi_1(A_R\cap A_L)\rightarrow \pi_1(A_L)$ is the map with $i_{LR}(a)=i_{LR}(b)=l$. Then the elements of $\Ker \Phi$ is generated by $rl^{-1}$, which means that we need to quotient $\pi_1(A_R)*\pi_1(A_L)$ with $r=l$. Then we get $\pi_1(A_R\cup A_L)\simeq \pi_1(X)\simeq \mathbb{Z}$.
\end{exmp}

\begin{exmp}[Two non-linked loops deleted from $\mathbb{R}^3$.] Now consider deleting two $S^1$ rings from $\mathbb{R}^3$, which are not linked. Putting $S^1$ on the $xy$ plane and splitting into right and left side as above gives $A_R$ and $A_L$, which can deformation retract to $S^1\vee S^1$, thus $\pi_1(A_R)\simeq \mathbb{Z}*\mathbb{Z}\simeq \langle r_1,r_2\rangle$ and $\pi_2(A_R)\simeq \mathbb{Z}*\mathbb{Z}\simeq \langle l_1,l_2\rangle$. Now the intersection $A_R\cap A_L$ can deformation retract to $S^1\vee S^1\vee S^1\vee S^1$, thus $\pi_1(A_R\cap A_L)\simeq \mathbb{Z}*\mathbb{Z}*\mathbb{Z}*\mathbb{Z}\simeq \langle a,b,c,d\rangle$. Now the inclusion map $i_{RL}:\pi_1(A_R\cap A_L)\rightarrow \pi_1(A_R)$ is the map with $i_{RL}(a)=i_{RL}(b)=r_1, i_{RL}(c)=i_{RL}(d)=r_2$, and $i_{LR}:\pi_1(A_R\cap A_L)\rightarrow \pi_1(A_L)$ is the map with $i_{LR}(a)=i_{LR}(b)=l_1, i_{LR}(c)=i_{LR}(d)=l_2$. Thus the elements of $\Ker \Phi$ is generated by $r_1l_1^{-1}$ and $r_2l_2^{-1}$, which means that we need to quotient  $\pi_1(A_R)*\pi_1(A_L)$ with $r_1=l_1, r_2=l_2$. Then we get $\pi_1(A_R\cup A_L)\simeq \pi_1(X)\simeq \mathbb{Z}*\mathbb{Z}$.

Indeed we can split this space into two spaces where each space contains one loop. Then the intersection is homeomorphic to $\mathbb{R}^3$, which is simply connected, hence we get $\mathbb{Z}*\mathbb{Z}$ again. 
\end{exmp}

\begin{exmp}[Loop and line deleted from $\mathbb{R}^3$.] Before deleting two linked $S^1$ rings from $\mathbb{R}^3$, first consider deleting $S^1$ and $\mathbb{R}$ piercing through $S^1$ from $\mathbb{R}^3$. Putting $S^1$ on the $xy$ plane and $\mathbb{R}$ at the center vertically, and splitting into right and left side gives $A_R$ and $A_L$, which can deformation retract to $S^1\vee S^1$, thus $\pi_1(A_R)\simeq \mathbb{Z}*\mathbb{Z}\simeq \langle r,r_{\mathbb{R}}\rangle$ and $\pi_2(A_R)\simeq \mathbb{Z}*\mathbb{Z}\simeq \langle l,l_{\mathbb{R}}\rangle$. Here the subscript ${}_\mathbb{R}$ implies the generator is a loop around vertical $\mathbb{R}$. Now the intersection $A_R\cap A_L$ can deformation retract to $S^1\vee S^1\vee S^1$, thus $\pi_1(A_R\cap A_L)\simeq \mathbb{Z}*\mathbb{Z}*\mathbb{Z}\simeq \langle a,b,c\rangle$. Here $c$ is the generator of loop around $\mathbb{R}$. Calculating the inclusion map here, however, needs some care. Since at the center of this space we have vertical $\mathbb{R}$ bar, we need to take the basepoint not at the center, but near toward one of subspaces. Take the basepoint closer to $A_R$. Then the inclusion map $i_{RL}:\pi_1(A_R\cap A_L)\rightarrow \pi_1(A_R)$ is the map with $i_{RL}(a)=i_{RL}(b)=r$ and $i_{RL}(c)=r_{\mathbb{R}}$. For the inclusion map $i_{LR}:\pi_1(A_R\cap A_L)\rightarrow \pi_1(A_L)$, the result is different, which is because $i_{RL}(a)\neq i_{RL}(b)$. Indeed, $i_{LR}(c)=l_{\mathbb{R}}$ and $l_{\mathbb{R}}^{-1}i_{LR}(a)l_{\mathbb{R}}=i_{LR}(b)$. We now may take $i_{LR}(a)=l$. This gives $\pi_1(X)\simeq \langle r, r_{\mathbb{R}},l,l_{\mathbb{R}}\rangle/\langle r_\mathbb{R}l_{\mathbb{R}}^{-1},rl^{-1},l_{\mathbb{R}}^{-1}ll_{\mathbb{R}}r^{-1}\rangle\simeq \langle r,R|R^{-1}rRr^{-1}\rangle$. Now notice that $R^{-1}rRr^{-1}=e$ implies $rR=Rr$, which means that we abelianize $\langle r,R\rangle\simeq \mathbb{Z}*\mathbb{Z}$. This is $\mathbb{Z}\times \mathbb{Z}$.

Indeed this space can be deformation retracted into the torus, $T$, which has a fundamental group $\pi_1(T)\simeq \mathbb{Z}$, and this confirms the above result.
\end{exmp}

\begin{exmp}[Two linked loops deleted from $\mathbb{R}^3$.] Finally we delete two linked $S^1$ rings from $\mathbb{R}^3$. Set one ring horizontally and split the space into left and right side, $A_R$ and $A_L$, as we have done in one ring case. Consider $A_R$ totally contains one another ring. Then we get $\pi_1(A_R)\simeq \mathbb{Z}\times \mathbb{Z}=\langle r,R\rangle/\langle rRr^{-1}R^{-1}\rangle$ and $\pi_1(A_L)\simeq \mathbb{Z}=\langle l\rangle$, and $\pi_1(A_R\cap A_L)\simeq \mathbb{Z}*\mathbb{Z}=\langle a,b\rangle$. Now $i_{RL}:\pi_1(A_R\cap A_L)\rightarrow \pi_1(A_R)$ is the map with $i_{RL}(a)=r, i_{RL}(b)=Rr^{-1}R^{-1}$, and $i_{LR}(a)=i_{LR}(b)=l$. Thus we get $\langle l,r,R\rangle/\langle rRr^{-1}R^{-1}, rl^{1-}\rangle\simeq \langle r,R\rangle/\langle rRr^{-1}R^{-1}\rangle\simeq \mathbb{Z}\times \mathbb{Z}$ again.

Indeed this space also can be deformation retracted into the torus, $T$, which confirms the above result.

Therefore, we have proven that the space with two linked loops and the space with two non-linked loops are not homeomorphic, because $\mathbb{Z}\times \mathbb{Z}\not\simeq \mathbb{Z}*\mathbb{Z}$.
\end{exmp}

\begin{lemma} For a bounded subspace $A$ of $\mathbb{R}^n$ with $n\geq 3$, $\pi_1(\mathbb{R}^n-A)\simeq \pi_1(S^n-A).$
\end{lemma}
\begin{proof}
Notice that $S^n$ can be thought as the one-point compactification of $\mathbb{R}^n$. Now we write $S^n-A$ as the union of $\mathbb{R}^n-A$ and an open ball $B$, where $B=\{\bullet\}\cup (\mathbb{R}^n-B_A)$, where $B_A$ is a closed ball containing $A$, which is possible to take since $A$ is bounded. Then $B$ is simply connected, and $B\cap (\mathbb{R}^n-A)\simeq S^{n-1}\times \mathbb{R}$ is also simply connected if $n\geq 3$, therefore by van Kampen's theorem we get the desired result.
\end{proof}

\begin{exmp}[Torus knot.] Take a relative prime positive primes $m,n$, we call the image of the embedding $f:S^1\rightarrow S^1\times S^1\subset \mathbb{R}^3$ defined as $f(z)=(z^m,z^n)$ a \textbf{torus knot} and write $K_{m,n}$. Now we want to calculate $\pi_1(\mathbb{R}^3-K_{m,n})$. Due to the lemma above, $\pi_1(\mathbb{R}^3-K_{m,n})\simeq \pi_1(S^3-K_{m,n})$.

Now notice that $S^3\simeq \partial D^4\simeq \partial(D^2\times D^2)\simeq \partial D^2\times D^2\cup D^2\times \partial D^2\simeq S^1\times D^2\cup D^2\times S^1$.\marginnote{\begin{lemma}$\partial(X\times Y)\simeq (\partial X\times \bar{Y})\cup (\bar{X}\times \partial Y)$.
\end{lemma}
\begin{proof}
Take $(x,y)\in \partial X\times \bar{Y}$, and let $N$ be the neighbor of $(x,y)$. By the definition of the product topology, we have open neighbor $U$ of $x$ and $V$ of $y$ such that $U\times V\subset N$. Since $x\in \partial X$, $U$ intersects with both $X$ and $X^c$. Also, since $y\in \bar{Y}$, the neighbor $V$ must contain the element in $Y$. Thus $U\times V$ contains the element in $X\times Y$ and $(X\times Y)^c$, and so $(x,y)\in \partial(X\times Y)$. Samely, if $(x,y)\in \bar{X}\times \partial Y$, then $(x,y)\in \partial(X\times Y)$.

Now take $(x,y)\in \partial(X\times Y)$. Suppose that $(x,y)\in (\partial X\times \bar{Y})^c\cap (\bar{X}\times \partial Y)^c$. If $x\notin \bar{X}$ then there is an open neighborhood of $x$ which does not contains any point of $X$, and same for $y$, $(x,y)\in \bar{X}\times \bar{Y}$. Therefore $x\notin \partial X$ and $y\notin \partial Y$. But then there is an open neighborhood of $x$ which does not contains any point of $X^c$, contradiction.
\end{proof}} Thus we can think $S^3$ as a union of two solid torus, one can be thought as the ordinary torus mapped into the $\mathbb{R}^3$ and the other can be thought as the closure of lefting which is one-point compactificated. Notice that the meridian circle of $S^1\times S^1$ bounds disk of first solid torus, and the longitudinal circle bounds disk of second solid torus. Denote the first solid torus as $T_i$ and second solid torus as $T_o$.

Now delete $K_{m,n}$ from $S^3$. This gives two spaces $T_i-K_{m,n}$ and $T_o-K_{m,n}$, whose union is $S^3-K_{m,n}$ and intersection is $S^1\times S^1-K_{m,n}$. Notice that $S^1\times S^1-K_{m,n}$ is path connected; indeed, if we shift $K_{m,n}$ a bit in $S^1\times S^1$, then we can deformation retract $S^1\times S^1-K_{m,n}$ into the shifted knot, which is homeomorphic to $S^1$. Therefore $\pi_1(S^1\times S^1-K_{m,n})\simeq \mathbb{Z}$. Also, since $T_{i,o}-K_{m,n}$ can be deformation retracted into the smaller torus, which also can be deformation retracted into a circle, $\pi_1(T_{i,o}-K_{m,n})\simeq \mathbb{Z}$. Denote $k$ be the generator of $\pi_1(S^1\times S^1-K_{m,n})$ and $a,b$ be the generator of $\pi_1(T_{i,o}-K_{m,n})$.

Now we need to think $i_{io}(k)$ and $i_{oi}(k)$. Indeed, $k$ can be represented as the loop $K_{m,n}$, therefore we need to calculate which represents the loop $K_{m,n}$ in $T_{i,o}$. For $T_i$, since meridian circle bounds disk, winding around meridian circle is meaningless in the sense of fundamental group of $T_i$. Therefore the only meaningful winding is winding around longitudinal circle, which means, $i_{io}(k)=a^m$. In $T_o$, the result is same except we exchange the role of meridian and longitudinal circle, which gives $i_{oi}(k)=b^n$. Therefore, $\pi_1(\mathbb{R}^3)\simeq \langle a,b|a^m=b^n\rangle\simeq \mathbb{Z}_m*\mathbb{Z}_n$ where $\mathbb{Z}_m\coloneqq \mathbb{Z}/m\mathbb{Z}$. Now since the abelianization of $\mathbb{Z}_m*\mathbb{Z}_n$ is $\mathbb{Z}_m\times \mathbb{Z}_n$\marginnote{\begin{lemma} If $G,H$ are abelian group, then the abelianization of $G*H$ is $G\times H$.\end{lemma}\begin{proof} By the abelianization, all the words becomes the form of $a^mb^n$ with $m,n\in\mathbb{Z}$. Take the map $\phi:(G*H)_{\textrm{ab}}\rightarrow G\times H$ as $\phi(a^mb^n)=(a^m,b^n)$. Then this is well defined, homomorphic, surjective. For injectivity, if $(a^m,b^n)=e$, then $a^m=e_G$ and $b^n=e_H$, thus $a^mb^n=e\in G*H$. Thus $\phi$ is isomorphism.
\end{proof}} and $\mathbb{Z}_m\times \mathbb{Z}_n\not\simeq\mathbb{Z}_k\times \mathbb{Z}_l$ if $\{m,n\}\neq \{k,l\}$, if we count the all the torus knots $K_{m,n}$ with different index has different knot group. 
\end{exmp}

\noindent\rule{\textwidth}{1pt}
\newline