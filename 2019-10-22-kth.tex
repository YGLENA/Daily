\mytitle{K-Theory}
\begin{exmp} Consider the set $\textrm{Vect}^1(B)$, the isomorphism classes of one-dimensional vector bundles over $B$. Then it is an abelian group with respect to the tensor product operation. The inverse of a line bundle $E\rightarrow B$ can be constructed by replacing its gluing matrices $g_{\beta\alpha}(x)\in GL_1(\mathbb{R})$ with their inverses: the cocycle condition is satisfied since $1\times 1$ matrices commute. Now if we give $E$ an inner product, we may rescale local trivializations $h_\alpha$ to be isometries, taking vectors in fibers of $E$ to vectors in $\mathbb{R}^1$ of the same length. Then all the values of the gluing functions $g_{\beta\alpha}$ are $\pm 1$, being isometries of $\mathbb{R}$. The gluing functions for $E\otimes E$ are the squares of these $g_{\beta\alpha}$'s, which is 1, thus $E\otimes E$ is the trivial line bundle, and so each element of $\textrm{Vect}^1(B)$ is its own inverse.\marginnote{The group $\textrm{Vect}^1(B)$ is isomorphic to the cohomology group $H^1(B;\mathbb{Z}_2)$ when $B$ is homotopy equivalent to a CW complex, which we will show later.}

Considering the complex vector bundles, we can get exactly same results, except the values of $g_{\beta\alpha}$'s are the complex numbers of norm $1$. Thus we cannot expect $E\otimes E$ to be trivial.\marginnote{The group $\textrm{Vect}_\mathbb{C}^1(B)$ is isomorphic to the cohomology group $H^2(B;\mathbb{Z})$, when $B$ is homotopy equivalent to a CW complex, which we will show later.}
\end{exmp}

\begin{defn} Consider a complex vector bundle $E\rightarrow B$ with gluing maps $g_{\beta\alpha}$. The \textbf{conjugate bundle} $\bar{E}\rightarrow B$ is a complex vector bundle defined by the conjugate of  gluing maps.
\end{defn}

\begin{exmp} For a complex vector line bundle $E\rightarrow B$. Then the complex line bundle $E\otimes \bar{E}$ is isomorphic to the trivial line bundle, since its gluing maps have values $z\bar{z}=1$ for $z$ a unit complex number.
\end{exmp}

\begin{defn} Consider a vector bundle $E\rightarrow B$. The \textbf{exterior power} $\wedge^k(E)$ is a vector bundle generated by the exterior product of gluing maps.
\end{defn}

\begin{defn} Let $F$ be a space. A \textbf{fiber bundle with fiber $F$} is a map $p:E\rightarrow B$ such that there is a cover of $B$ by open sets $U_\alpha$ for each of which there exists a homeomorphism $h_\alpha:p^{-1}(U_\alpha)\rightarrow U_\alpha\times F$ taking $p^{-1}(b)$ to $\{b\}\times F$ for each $b\in U_\alpha$.
\end{defn}

\begin{exmp}
~\begin{enumerate}
\item For a vector bundle $E$ with an inner product, we can define a subspace $S(E)$ consisting of the unit spheres in all the fibers. Then the natural projection $S(E)\rightarrow B$ is a fiber bundle with sphere fibers, which is called a \textbf{sphere bundle}.
\item Similarly, we can define a \textbf{disk bundle} $D(E)$ consisting of the unit disks in all the fibers.
\item Indeed, we can define $S(E)$ and $D(E)$ without an inner product. For $S(E)$, we take the quotient of the complement of the zero section in $E$ obtained by identifying each nonzero vector with all positive scalar multiples of itself. For $D(E)$, consider the mapping cylinder of the projection $S(E)\rightarrow B$, the quotient space of $S(E)\times[0,1]$ obtained by identifying two points in $S(E)\times \{0\}$ if they have the same image in $B$.
\item The canonical line bundle $E\rightarrow \mathbb{R}P^n$ has its unit sphere bundle $S(E)$ as the space of unit vectors in lines through the origin in $\mathbb{R}^{n+1}$. Since each unit vector uniquely determines the line containing it, $S(E)$ is $S^n$.
\item For a vector bundle $E\rightarrow B$, the \textbf{projective bundle}, $P(E)\rightarrow B$, is the fiber bundle where $P(E)$ is the space of all lines through the origin in all the fibers of $E$. We topologize $P(E)$ as the quotient of the sphere bundle $S(E)$ obtained by factoring out scalar multiplication in each fiber. Then the fibers are homeomorphic to $\mathbb{R}P^{n-1}$.
\item For an $n$-dimensional vector bundle $E\rightarrow B$, the \textbf{flag bundle} $F(E)\rightarrow B$ is a vector bundle with $F(E)$ the subspace of the $n$-fold product of $P(E)$, consisting of $n$-tubles of orthogonal lines in fibers of $E$. For any $k\leq n$, one could also take $k$-tuples of orthogonal lines in fibers of $E$, and get a bundle $F_k(E)\rightarrow B$.
\item The \textbf{Stiefel bundle} $V_k(E)\rightarrow B$ is a refinement of the flag bundle, that is, the points of $V_k(E)$ are $k$-tuples of orthogonal unit vectors in fibers of $E$.
\item The \textbf{Grassmann bundle} $G_k(E)\rightarrow B$, generalization of projective bundle, is the quotient space of Stiefel bundle $V_k(E)$, obtained by identifying two $k$-frames in a fiber if they span the same subspace of the fiber.
\end{enumerate}
\end{exmp}
\noindent\rule{\textwidth}{1pt}
\newline