\mytitle{An introduction to homological algebra}
\begin{exer} For a category $\mathsf{Ch}$, and $f$ be its morphism, show that the complex $\Ker(f)$ is a kernel of $f$ and the complex $\Coker(f)$ is a cokernel of $f$.
\end{exer}
\begin{solution} Let $i_n:\Ker(f)_n\rightarrow C_n$ be the kernel of $f_n:C_n\rightarrow D_n$. Then we have the universal properties for each components: for any $g_n:B_n\rightarrow C_n$ such that $f_n\circ g_n=0$, there is a unique morphism $u_n:B_n\rightarrow C_n$ such that $i_n\circ u_n=g_n$. Now suppose that $\{B_n\}$ is a chain complex and $\{g_n\}$ is a chain map. What we now need to show is $\{u_n\}$ is a chain map, that is, $d\circ u_n=u_{n-1}\circ d$. Now notice that $i_{n-1}\circ u_{n-1}\circ d=g_{n-1}\circ d=d\circ g_n=d\circ i_n\circ u_n=i_{n-1}\circ d\circ u_n$, and by previous proposition we know that the kernel $i_{n-1}$ is monic, thus $u_{n-1}\circ d=d\circ u_n$.
\end{solution}

\begin{defn} An \textbf{abelian category} is an additive category $\mathsf{A}$ such that
\begin{enumerate}
\item every map in $\mathsf{A}$ has a kernel and cokernel;
\item every monic in $\mathsf{A}$ is the kernel of its cokernel;
\item every epi in $\mathsf{A}$ is the cokernel of its kernel.
\end{enumerate}
\end{defn}

\begin{exmp} The category $R-\mathsf{mod}$ is an abelian category. Indeed, every morphism $f:c\rightarrow d$ has a kernel $\Ker(f)=\{x\in c:f(x)=0\}\hookrightarrow c$ and a cokernel $\Coker(f)=d\rightarrow d/\Ima(f)$. For monic $f$, $\Ima(f)\simeq c$, thus $\Coker(f)=d\rightarrow d/c$ and its kernel is $c$. For epic $f$, the cokernel of $\Ker(f)$ is $c\hookrightarrow c/\Ker(f)$, which is surjective and has a same structure with $f:c\rightarrow d$, thus it is $f$.
\end{exmp}

\begin{defn} For an abelian category $\mathsf{C}$ and its morphism $f$, the \textbf{image} of a map $f:b\rightarrow c$ is the subobject of $c$ defined as $\Ker(\Coker(f))$.\marginnote{This definition is same with our previous definition of $\Ima$ in $R-\mathsf{mod}$, because $\Ker(\Coker(f))=\Ker(c\rightarrow c/\Ima(f))=\Ima(f)$.}
\end{defn}

\begin{prop} For an abelian category $\mathsf{A}$, every morphism $f:b\rightarrow c$ factors as
\begin{equation}
b\xrightarrow{e}\Ima(f)\xrightarrow{m} c
\end{equation}
where $e=\Coker(\Ker(f))$ is an epimorphism and $m=\Ker(\Coker(f))$ is a monomorphism.
\end{prop}
\begin{proof}
Take $m=\Ker(\Coker(f))$, which is monic since it is a kernel. Since $\Coker(f)\circ f=0$ by definition, $f$ factors as $f=m\circ e$ for some unique $e$, which is epic.\marginnote{The argument of this statement is in Categories for the working mathematician, S. MacLane, p189.} Now for any $g:a\rightarrow b$, $f\circ g=0$ if and only if $e\circ g=0$, since $m$ is monic. Thus $\Ker(f)=\Ker(e)$. But since $e$ is epic, $e=\Coker(\Ker(e))=\Coker(\Ker(f))$.
\end{proof}

\begin{defn} For an abelian category $\mathsf{A}$, a sequence $a\xrightarrow{f} b\xrightarrow{g} c$ is \textbf{exact} if $\Ker(g)=\Ima(f)$.
\end{defn}

\begin{defn} For an abelian category $\mathsf{A}$, the category $\mathsf{Ch}(\mathsf{A})$ is a category whose objects are chain complexes in $\mathsf{A}$ and morphisms are chain maps in $\mathsf{A}$.
\end{defn}

\begin{thm} For an abelian category $\mathsf{A}$, the category $\mathsf{Ch}(\mathsf{A})$ is an abelian category.
\end{thm}
\begin{proof}
The argument for showing additive category is same with $R-\mathsf{mod}$ case. The argument for the first condition is just same with the case $\mathsf{Ch}$. For the second and third condition, consider the components of the morphism $f$, which are all monic(epic) if and only if $f$ is monic(epic). Since $\mathsf{A}$ is an abelian category, the components of $f_n$ is the kernel of its cokernel(cokernel of its kernel), thus $f$ also is.
\end{proof}

\begin{exer} Show that a sequence $0\rightarrow A_\bullet\xrightarrow{f_\bullet} B_\bullet \xrightarrow{g_\bullet} C_\bullet\rightarrow 0$ of chain complexes is exact in $\mathsf{Ch}$ just in case each sequence $0\rightarrow A_n\xrightarrow{f} B_n\xrightarrow{g} C_n\rightarrow 0$ is exact in $\mathsf{A}$.
\end{exer}
\begin{solution} What we need to show is $\Ker(g_\bullet)=\Ima(f_\bullet)$, which is equivalent to $\Ker(g_n)=\Ima(g_n)$ for all $n$.
\end{solution}

\begin{exmp} A \textbf{double complex} or \textbf{bicomplex} in $\mathsf{A}$ is a family $\{C_{p,q}\}$ of objects in $\mathsf{A}$, together with maps $d^h:C_{p,q}\rightarrow C_{p-1,q}$ and $d^v=C_{p,q}\rightarrow C_{p,q-1}$ such that $d^h\circ d^h=d^v\circ d^v=d^v\circ d^h+d^h\circ d^h\circ d^v=0$. If there are finitely many nonzero $C_{p,q}$ along each diagonal line $p+q=n$, then we call $C$ \textbf{bounded}.

Due to the anticommutivity, the maps $d^v$ are not maps in $\mathsf{Ch}$, but the chain maps $f_{\bullet,q}:C_{\bullet,q}\rightarrow C_{\bullet,q-1}$ can be defined by introducing
\begin{equation}
f_{p,q}=(-1)^p d_{p,q}^v:C_{p,q}\rightarrow C_{p,q-1}
\end{equation}
\end{exmp}

\begin{exmp}[Total complexes] For a bicomplex $C$, we define the \textbf{total complexes} $\textrm{Tot}^{\prod}(C)$ and $\textrm{Tot}^{\oplus}(C)$ as
\begin{equation}
\textrm{Tot}^{\prod} (C)_n=\prod_{p+q=n}C_{p,q},\quad \textrm{Tot}^\oplus (C)_n=\oplus_{p+q=n}C_{p,q}
\end{equation}
Then $d=d^h+d^v$ defines maps $d:\textrm{Tot}^{\prod,\oplus}(C)_\rightarrow \textrm{Tot}^{\prod,\oplus}(C)_{n-1}$

such that $d\circ d=0$ since $d^h\circ d^v+d^v\circ d^h=0$, thus they are chain complexes. Notice that the total complexes does not always exists, because the infinite (co)direct products could not exists. An abelian category is \textbf{(co)complete} if all (co)direct products exist. Both $R-\mathsf{mod}$ and $\mathsf{Ch}(R-\mathsf{mod})$ are complete and cocomplete.
\end{exmp}

\begin{exer} For a bounded double complex $C$ with exact rows(columns), show that $\textrm{Tot}^{\prod}(C)=\textrm{Tot}^{\oplus}(C)=\textrm{Tot}(C)$ is acyclic.
\end{exer}
\begin{solution} Since $C$ is bounded, we can write the element of $\textrm{Tot}(C)$ as $c=(\cdots,0,c_{0,0},c_{1,-1},\cdots,c_{k,-k},0,\cdots)$, by some shifting of indexes if needed. Suppose that $d(c)=0$, which means,
\begin{equation}
(\cdots,0,d^v(c_{0,0}),d^v(c_{1,-1})+d^h(c_{0,0}),\cdots,d^h(c_{k,-k}),0,\cdots)=0
\end{equation}
Now we want to find the element $b$ of $\textrm{Tot}(C)$ such that $d(b)=c$. Without loss of generality, we may let the columns are exact. Then since $d^v(c_{0,0})=0$, there is $b_{1,0}$ such that $d^v(b_{1,0})=c_{0,0}$. Now then we have
\begin{equation}
d^v(c_{1,-1})+d^h(d^v(b_{1,0}))=d^v(c_{1,-1})-d^v(d^h(b_{1,0}))=d^v(c_{1,-1}-d^h(b_{1,0}))=0
\end{equation}
and due to the exactness we have $b_{1,-1}$ such that $d^v(b_{1,-1})=c_{1,-1}-d^h(b_{1,0})$. By doing this inductively, which has finitely many steps because $C$ is bounded.
\end{solution}

\begin{exer} Give examples of
\begin{enumerate}
\item a second quadrant double complex $C$ with exact columns such that $\textrm{Tot}^{\prod}(C)$ is acyclic but $\textrm{Tot}^{\oplus}(C)$ is not;
\item a second quadrant double complex $C$ with exact rows such that $\textrm{Tot}^{\oplus}(C)$ is acyclic but $\textrm{Tot}^{\prod}(C)$ is not;
\item a double complex in the entire plane for which every row and every column is exact, yet neither $\textrm{Tot}^{\prod}(C)$ nor $\textrm{Tot}^{\oplus}(C)$ is acyclic.
\end{enumerate}
\begin{solution}
~\begin{enumerate}
\item Consider the following double complex.
\begin{equation}
\begin{tikzcd}
\ddots\arrow{d}{1}& &\\
\mathbb{Z}&\mathbb{Z} \arrow{l}{\times 2} \arrow{d}{1}&\\
&\mathbb{Z}&\mathbb{Z} \arrow{l}{\times 2}
\end{tikzcd}
\end{equation}
Here all the non-represented objects are zero objects and morphisms are zero morphisms. Notice that the columns are exact. Now notice that this double complex takes
\begin{equation}
(\cdots,a_{-2},a_{-1},a_0)\mapsto (\cdots,a_{-2}+2a_{-1},a_{-1}+2a_0)
\end{equation}
For $\textrm{Tot}^{\prod}(C)$, take $(\cdots,4,-2,1)$. This is in the kernel of above map, but not in the image of zero map. For $\textrm{Tot}^{\oplus}(C)$, we get the number $n$ such that $a_{-k}=0$ for all $k>n$. Now since 
\begin{equation}
(a_{-n},\cdots,a_0)\mapsto (2a_{-n},\cdots,a_{-1}+2a_0)
\end{equation}
thus if $(a_{-n},\cdots,a_0)$ is the kernel of above map then $a_{-n}=0$, and inductively all $a_0=0$.
\item Consider the following double complex.
\begin{equation}
\begin{tikzcd}
\ddots&\mathbb{Z}\arrow{l}{1}\arrow{d}{1}&\\
&\mathbb{Z}&\mathbb{Z}\arrow{l}{1}
\end{tikzcd}
\end{equation}
Here all the non-represented objects are zero objects and morphisms are zero morphisms. Notice that the rows are exact. Now notice that this double complex takes
\begin{equation}
(\cdots,a_{-2},a_{-1},a_0)\mapsto (\cdots,2a_{-2}+a_{-1},2a_{-1}+a_{0},a_{0})
\end{equation}
For $\textrm{Tot}^{\oplus}(C)$, suppose that we have $(\cdots,2a_{-2}+a_{-1},2a_{-1}+a_{0},a_0)=(\cdots,0,0,1)$. Then $a_0=1$, thus $a_{-1}=-2$, and $a_{-2}=-4$, and so on so we get $a_{-n}=2^n$, which is not in $\textrm{Tot}^{\oplus}(C)$, thus $(\cdots,0,0,1)$ is not in the image of the above map, but in the kernel of zero map. For $\textrm{Tot}^{\prod}(C)$, for $(\cdots,b_{-2},b_{-1},b_0)$, we take $a_0=b_0$ and $a_{-n}=b_{-n}-2a_{-(n-1)}$, which is well defined for all $n$. 
\item Consider the following double complex.
\begin{equation}
\begin{tikzcd}
\ddots\arrow{d}{-1}& &\\
\mathbb{Z}&\mathbb{Z} \arrow{l}{1} \arrow{d}{-1}&\\
&\mathbb{Z}&\mathbb{Z} \arrow{l}{1}\arrow{d}{-1}\\
&&\ddots
\end{tikzcd}
\end{equation}
Here all the non-represented objects are zero objects and morphisms are zero morphisms. Notice that the rows and columns are exact. Now notice that this double complex takes
\begin{equation}
(\cdots,a_{-1},a_0,a_{-1},\cdots)\mapsto (\cdots,-a_{-1}+a_0,-a_0+a_1,\cdots)
\end{equation}
For $\textrm{Tot}^{\prod}(C)$, $(\cdots,1,1,\cdots)$ is in the kernel of above map, but not in the image of zero map. For $\textrm{Tot}^{\oplus}(C)$, if $(\cdots,-a_{-1}+a_0,-a_0+a_1,-a_1+a_2,\cdots)=(\cdots,0,1,0,\cdots)$ then we get $\cdots=a_{-1}+1=a_0+1=a_1=a_2=\cdots$, which is not in $\textrm{Tot}^{\oplus}(C)$, thus $(\cdots,0,1,0,\cdots)$ is not in the image of the above map, but in the kernel of zero map.
\end{enumerate}
\end{solution}
\end{exer}

\begin{defn} Let $C$ be a chain complex and $n$ be an integer. The complex $\tau_{\geq n}C$ defined by
\begin{equation}
(\tau_{\geq n}C)_i=\begin{cases}
0, &i<n\\
Z_n, & i=n\\
C_i,&i>n
\end{cases}
\end{equation}
is called the \textbf{truncation of $C$ below $n$}. Notice that
\begin{equation}
H_i(\tau_{\geq n}C)=\begin{cases}
0,&i<n\\
H_i(C),&i\geq n
\end{cases}
\end{equation}
The quotient $\tau_{<n}C=C/(\tau_{\geq n}C)$ is called the \textbf{truncation of $C$ above $n$}. Notice that
\begin{equation}
H_i(\tau_{< n}C)=\begin{cases}
H_i(C),&i<n\\
0,&i\geq n
\end{cases}
\end{equation}
The complex $\sigma_{<n}C$ defined by
\begin{equation}
(\sigma_{<n}C)_i=\begin{cases}
C_i,&i<n\\
0,&i\geq n
\end{cases}
\end{equation}
is called the \textbf{brutal truncation of $C$ above $n$}. Notice that
\begin{equation}
H_i(\tau_{\geq n}C)=\begin{cases}
H_i(C),&i<n\\
0,&i> n\\
C_n/B_n,&i=n
\end{cases}
\end{equation}
The quotient $\sigma_{\geq n}C=C/(\sigma_{<n}C)$ is called the \textbf{brutal truncation of $C$ below $n$}. Notice that
\begin{equation}
H_i(\tau_{\geq n}C)=\begin{cases}
0,&i<n\\
H_i(C),&i> n\\
C_n/B_n,&i=n
\end{cases}
\end{equation}
\end{defn}

\begin{defn} If $C$ is a chain complex and $p$ is an integer, we take a new complex $C[p]$ defined as
\begin{equation}
C[p]_n=C_{n+p}
\end{equation}
with differential $(-1)^p d$. If $C$ is a cochain complex, we take
\begin{equation}
C[p]^n=C^{n-p}
\end{equation}
with differential $(-1)^p d$. This job is called \textbf{shifting indices} or \textbf{translation}. We call $C[p]$ the \textbf{$p$-th translate of $C$}. Notice that
\begin{equation}
H_n(C[p])=H_{n+p}(C),\quad H^n(C[p])=H^{n-p}(C)
\end{equation}
for chain and cochain complex respectively.

For a (co)chain map $f:C\rightarrow D$, we define $f[p]:C[p]\rightarrow D[p]$ as
\begin{equation}
f[p]_n=f_{n+p}, \quad f[p]^n=f^{n-p}
\end{equation}
for chain and cochain map respectively. This makes translation a functor.
\end{defn}

\begin{exer} If $C$ is a complex, show that there are exact sequences of complexes:
\begin{equation}
0\rightarrow Z(C)\rightarrow C\xrightarrow{d} B(C)[-1]\rightarrow 0
\end{equation}
\begin{equation}
0\rightarrow H(C)\rightarrow C/B(C)\xrightarrow{d} Z(C)[-1]\rightarrow H(C)[-1]\rightarrow 0
\end{equation}
\end{exer}
\begin{solution}
We can expand the first sequence as
\begin{equation}
\begin{tikzcd}
&\vdots\arrow{d}&\vdots\arrow{d}&\vdots\arrow{d}&\\
0\arrow{r}&Z_{n+1}\arrow[r,hook] \arrow{d}{d_{n+1}}&C_{n+1}\arrow{r}{d_{n+1}}\arrow{d}{d_{n+1}}&B_{n}\arrow{d}{d_{n}}\arrow{r}&0\\
0\arrow{r}&Z_{n}\arrow[r,hook] \arrow{d}{d_{n}}&C_{n}\arrow{r}{d_{n}}\arrow{d}{d_{n}}&B_{n-1}\arrow{d}{d_{n-1}}\arrow{r}&0\\
0\arrow{r}&Z_{n-1}\arrow[r,hook] \arrow{d}{d_{n-1}}&C_{n-1}\arrow{r}{d_{n-1}}\arrow{d}{d_{n-1}}&B_{n-2}\arrow{d}{d_{n-2}}\arrow{r}&0\\
&\vdots&\vdots&\vdots&
\end{tikzcd}
\end{equation}
which commutes, and all the rows are exact by first isomorphism theorem, thus the sequence is exact. Similarly, we can expand the second sequence as the sequence of
\begin{equation}
0\rightarrow H_n\xhookrightarrow{i_n} C_n/B_n\xrightarrow{d_n} Z_{n-1}\xrightarrow{q_{n-1}} H_{n-1}\rightarrow 0
\end{equation}
which is exact since $\Ima i_n=H_n=Z_n/B_n=\Ker d_n$ and $\Ima d_n=B_{n-1}=\Ker q_{n-1}$.
\end{solution}

\noindent\rule{\textwidth}{1pt}
\newline