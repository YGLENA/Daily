\mytitle{Chern-Simons Theory}
\begin{defn} A \textbf{fiber bundle} is a structure $(E,B,\pi,F)$ where $E,B,F$ are topological spaces and $\pi:E\rightarrow B$ is a continuous surjective map satisfying for every $x\in E$ there is an open neighborhood $U\subset B$ of $\pi(x)$ such that there is a homeomorphism $\phi:\pi^{-1}(U)\rightarrow U\times F$ such that $\textrm{proj}_1\circ\phi=\pi|_{\pi^{-1}(U)}$.
\end{defn}

\begin{defn} A \textbf{topological group} $G$ is a topological space which is also a group such that the product map $(x,y)\mapsto xy$ and inverse map $x\mapsto x^{-1}$ are continuous.
\end{defn}

\begin{defn} Let $M$ be a manifold and $G$ be a topological group. A \textbf{principal $G$-bundle} is a fiber bundle $\pi:E\rightarrow M$ with a continuous right action $E\times G\rightarrow E$ such that
\begin{enumerate}
\item $G$ preserves the fibers of $P$,
\item $G$ acts on $E$ freely and transitively.
\end{enumerate}
\end{defn}

\begin{defn} Let $(E,M)$ and $(E',M')$ are principal $G$-bundles. If a map $\phi:E\rightarrow E'$ is a $C^\infty$ map which commutes with right actions, then we call $\phi$ a \textbf{bundle map}.
\end{defn}

\begin{prop} Let $(E,M)$ and $(E',M')$ are principal $G$-bundles and $\phi:E\rightarrow E'$ is a bundle map. Then there is a unique map $f:M\rightarrow M'$ satisfying $\pi'\circ\phi=f\circ \pi$.
\end{prop}
\begin{proof}
Later.
\end{proof}


\noindent\rule{\textwidth}{1pt}
\newline