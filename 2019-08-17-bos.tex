\mytitle{Quantum Physics in one dimension}
\begin{statement}{Free electron gas} For the free electron gas with temperature $T=0$, the free electron occupation can be written as
\begin{equation}
n_k=\begin{cases}
1,&0\leq k\leq k_F\\
0,&k_F<k
\end{cases}
\end{equation}
which has a discontinuity at the \textbf{fermi surface} $k_F$ with amplitude $1$. The excitations of the state consist in adding particles with a well-defined momentum $k$, which has energy $\epsilon(k)$. The probability to find a state with a frequency $\omega$ and a momentum $k$ can be written as the \textbf{spectral function} $A(k,\omega)$, which is $\delta(\omega-\xi(k))$ in free electron gas where $\xi(k)=\epsilon(k)-\mu$.
\end{statement}

\begin{statement}{Fermi liquid theory}
\end{statement}
\noindent\rule{\textwidth}{1pt}
\newline