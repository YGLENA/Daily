\mytitle{K-Theory of $C^*$-algebras in solid state physics}
\marginnote{This is the part of the digitalized version of the Bellissard's article written on 1986.} A real sample used in experimental solid state physics is made of a finite assembly of atoms bound together by electrostatic interactions. However, even though finite and relatively small(few millimeters in size, sometimes smaller in crystallography), any sample contains so many atoms that it is better described via the use of the infinite volume limit. On the other hand, it appears homogeneous at large scale while at atomic scale the disorder breaks any translation invariance.

When studying electronic properties of such a crystal, it is commonly admitted that a one electron approximation is usually quite good. Collective properties of the electrons gas are described through the model of a perfect Fermi gas. The disorder appears simply as external forces coming from the random positions of the atoms or impurities. In this set up, the Schr{\"o}dinger operator for an electron, acting on the space $L^2(\mathbb{R}^D)$($D$ being the dimension of the crystal), is given by an effective Hamiltonian of the form:
\begin{equation}
H=-\frac{\hbar^2}{2m}\nabla^2+\sum_i V_i(x-x_i)
\end{equation}
where $V_i$ is the potential created by the $i$-th atom or impurity and $x_i$ denotes its position.

The disorder may have several sources. One is given by the randomness of the position of the atoms. This randomness may come from the occurrence of many defects due to the way the sample has been prepared. It may as well come from structural reasons, namely the thermodynamical equilibrium favors the occurence of some disorder: this is the case for amorphous materials(glasses) or quasi-crystals. Another source of disorder is also given by the impurities which modify the atomic potential at random positions.

In any case, to describe such a system in full generality, one usually introduces a probability space $(\Omega, \Sigma, \mu)$, the points of which labelling the Hamiltonian and describing in an implicit way the configuration of the material. In other words $H$ becomes a function of $\omega\in \Omega$. For obvious mathematical reasons, one demands that $H$ be a measurable function of $\omega$l at least in the strong resolvent sense(we recall that Borel sets are same for the norm and the strong topology in the algebra of bounded operator in a separable Hilbert space). To describe the macroscopic homogeneity of the material we just remark that translating the electron in the sample is equivalent to translating the atoms backward and since the sample looks almost the same at any place, this is just changing the configuration $\omega$ in $\Omega$. Therefore, there must be an action $\omega\rightarrow T_x\omega$ of the translation group $\mathbb{R}^D$ on $\Omega$ such that, if $U(x)$ is the unitary operator representing the translation by $x$ in the Hilbert space $L^2(\mathbb{R}^D)$ then
\begin{equation}
U(x)H_\omega U(x)^\dagger = H_{T_x\omega}
\label{eq:cov}
\end{equation}
This action will be at least measurable and will satisfy the group property $T_{x}T_y=T_{x+y}$. As we shall see in the end of this section, we may assume $\Omega$ to be actually a compact space and the $\mathbb{R}^D$ action to be given by a group of homeomorphisms. The probability measure will serve later on in dealing with "self averaging" quantities.

The framework can be applied to the study of the phonon spectrum as well. For indeed, phonons are elementary mechanical exciatations of the atoms around their equilibrium position. In the approximation where there is no phonon-phonon interaction, they are described quantum mechanically as a free Bose field, and only the one phonon Hamiltonian must be considered: we must solve the corresponding classical problem. The classical eigenmodes are just eigenvalues of a discrete Schr{\"o}dinger-like operator with random coefficients. Again the randomness comes from the disorder inside the sample. In any case this operator will obey a covariance condition given by \ref{eq:cov}.

Even though in practice a sample is given from the beginning and will not change at low temperature during the experiment, the physical energy operator is not the operator $H_\omega$ alone corresponding to the given configuration of the disorder. Otherwise it would imply the choice of an origin of coordinates inside the sample, preventing the use of its homogeneity: there is no physical reason "a priori" to prefer one origin instead of one another! Therefore the observable algebra must contain not only $H_\omega$ but also the whole family of its translated. Since in general two elements of this family do not commute \textbf{the observable algebra will be non commutative in a fundamental way.}

When performing calculations in quantum mechanics, we need to compute several operators starting from the basic observables. In particular, we will need operators defined through series expansion and the question of convergence will be addressed. Therefore we need to define on the set of observables an algebraic structure which will allow us to make computations, and also a topology which will be essential in proving convergence of infinite series. Clearly the algebra obtained will be a $*$-algebra since we need to distinguish real numbers from the complex ones. However there is a wide choice of possible topologies depending upon the technical point of view we will choose. Nevertheless there is a canonical choice namely \textbf{a topology which is of purely algebraic origin.} There are two kinds of such topological $*$-algebras: the $C^*$-algebras and $W^*$-algebras. For in the former case the norm of an element $A$ is nothing but the spectral radius(a purely algebraic object) of $A^* A$. On the other hand a $W^*$-algebra is a $C^*$-algebra with a predual(which is unique). Equivalently it is a weakly closed $*$-subalgebra of $B(H)$. The von Neumann theorem, shows that it coincides with its bicommutant(again an algebraic property even though it may depend upon the representation). As we shall see the $W^*$-algebra built out of the translated of the original Hamiltonian is too large to contain any relevant informations. For this reason we shall prefer the $C^*$-algebra generated by the Hamiltonian and its translated. In a sense this algebra is the smallest object which contains all the relevant physical informations.

The next step in describing the formal framework consists in answering the question whether it is possible to recover the nature of the probability space $\Omega$ from the physical data. Since the translated of the Hamiltonian must belong to the algebra of observables, it is natural to built $\Omega$ out of the familly of self adjoint operators we obtain in this way. For this reason we introduce the following mathematical criterion to describe what we call "homogeneity".
\begin{defn} A bounded operator $A$ on $L^2(\mathbb{R}^D)$ is called \textbf{homogeneous} if the family of its translated $\{U(x)AU(x)^*:x\in \mathbb{R}^D\}$ has a compact strong closure.
\end{defn}


\noindent\rule{\textwidth}{1pt}
\newline