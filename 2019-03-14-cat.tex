\mytitle{Category theory in context}
\begin{exmp}
~\begin{enumerate}
\item For vector space of any dimension over the field $\mathbb{K}$, the map $\mathrm{ev}:V\rightarrow V^{**}$ that sends $v\in V$ to $\mathrm{ev}_v:V^*\rightarrow \mathbb{K}$ defines the components of a natural transformation from the identity endofunctor on $\mathsf{Vect}_{\mathbb{K}}$ to the double dual functor. The map $V\xrightarrow{\phi}W$ becomes $V\xrightarrow{\phi}W$ by the identity endofunctor and $V^{**}\xrightarrow{\phi^{**}}W^{**}$ by the double dual functor. Since $\mathrm{ev}_{\phi (v)}=\phi^{**}(\mathrm{ev}_v)$, this is natural transformation. However, there is no natural isomorphism between the identity functor and its dual functor on finite-dimensional vector spaces, which is because the identity functor is covariant but the dual functor is contravariant.
\item Consider $\mathsf{cHaus}$ as a category of compact Hausdorff spaces and continuous maps, and $\mathsf{Ban}$ as a category of Banach spaces and continuous linear maps. Consider a finite signed measure $\mu:\textrm{Baire}(X)\rightarrow \mathbb{R}$ where $\textrm{Baire}(X)$ is a Baire algebra of $X$, a $\sigma$-algebra generated by closed $G_\delta$ sets. The Jordan decomposition of $\mu$ gives $\mu=\mu_++\mu_-$, which gives the norm $\|\mu\|=\mu_+(X)+\mu_-(X)$, and this gives the Banach space of a finite signed Baire measure $\Sigma(X)$. Then we can define a functor $\Sigma:\mathsf{cHaus}\rightarrow \mathsf{Ban}$, which takes a continuous map $f:X\rightarrow Y$ to the map $\Sigma(f)(\mu)=\mu\circ f^{-1}:\Sigma (X)\rightarrow \Sigma(Y)$. Also, consider a functor $C^*:\mathsf{cHaus}\rightarrow \mathsf{Ban}$, which takes $X$ to the linear dual $C(X)^*$ of the Banach space $C(X)$ of continuous real-valued functions on $X$.

Now for each $\mu\in \Sigma(X)$, there is a linear functional $\phi_\mu:C(X)\rightarrow \mathbb{R}$, which is defined as $\phi_\mu(g)=\int_X g d\mu$ for $g\in C(X)$. Now for each $\mu\in \Sigma(X),f:X\rightarrow Y, h\in C(Y)$, since $\int_X h\circ f d\mu=\int_Y h d(\mu\circ f^{-1})$, which shows that the morphisms $\mu\mapsto \phi_\mu$ are the components of the natural transformation $\eta:\Sigma\rightarrow C^*$. Furthermore, the \textbf{Riesz representation theorem} says that this is a natural isomorphism.

\item Consider a category of commutative ring $\mathsf{cRing}$ and a category of group $\mathsf{Group}$. For a commutative ring $K$, consider the general linear group $GL_n K$ and the group of units $K^*$. Then $GL_n$ and $(-)^*$ are functors. Now for each general linear group $M$ consider the determinant $\det_n M$. Since $M$ is invertible, $\det_n M\in K^*$. Furthermore, for any ring homomorphism $\phi:K\rightarrow K'$, $\det_{K'}\circ GL_n(\phi)=\phi^*\circ \det_{K}$, thus the morphisms $\det_K$ are the components of the natural transformation $\det:GL_n\rightarrow (-)^*$.
\end{enumerate}
\end{exmp}
\noindent\rule{\textwidth}{1pt}
\newline