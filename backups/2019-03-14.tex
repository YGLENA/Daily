\mytitle{Category theory in context}
\begin{exmp}
~\begin{enumerate}
\item For vector space of any dimension over the field $\mathbb{K}$, the map $\mathrm{ev}:V\rightarrow V^{**}$ that sends $v\in V$ to $\mathrm{ev}_v:V^*\rightarrow \mathbb{K}$ defines the components of a natural transformation from the identity endofunctor on $\mathsf{Vect}_{\mathbb{K}}$ to the double dual functor. The map $V\xrightarrow{\phi}W$ becomes $V\xrightarrow{\phi}W$ by the identity endofunctor and $V^{**}\xrightarrow{\phi^{**}}W^{**}$ by the double dual functor. Since $\mathrm{ev}_{\phi (v)}=\phi^{**}(\mathrm{ev}_v)$, this is natural transformation. However, there is no natural isomorphism between the identity functor and its dual functor on finite-dimensional vector spaces, which is because the identity functor is covariant but the dual functor is contravariant.
\item Here is one interesting relation with physics: define an endofunctor of $\mathsf{Vect}_{\mathbb{K}}$ by $V\mapsto V\otimes V$. Then we have a natural transformation between this endofunctor and identity functor, which is zero map. 
\end{enumerate}
\end{exmp}
\noindent\rule{\textwidth}{1pt}
\newline