\mytitle{Algebraic Topology}
\begin{lemma} For a covering space and map $p:\tilde{X}\rightarrow X$ of $X$, if $U$ is evenly covered open set and $W\subset U$ is also an open set, then $W$ is also evenly covered.
\end{lemma}
\begin{proof}
By the definition of evenly covered open set, we have a collection of open sets $p^{-1}(U)=\cup_{\alpha}U_\alpha$ where $p|_{U_\alpha}$ is homeomorphism. Now we may write $p^{-1}(W)=\cup_{\alpha}\left(U_\alpha\cap p^{-1}(W)\right)$, using de Morgan's law. Writing $U_\alpha\cap p^{-1}(W)=W_\alpha$, we can show that $W_\alpha\cup W_\beta=\emptyset$ if $\alpha\neq \beta$. Now since $p$ is homeomorphism, the restriction of $p$ on $W_\alpha$ is homeomorphism, and $p(W_\alpha)=p(U_\alpha\cap p^{-1}(W))=p(U_\alpha)\cap p(p^{-1}(W))=p(U_\alpha)\cap W=U\cap W=W$.\marginnote{For any map $f:X\rightarrow Y$, if $f$ is surjective and $B\subset Y$, then $f(f^{-1}(B))=B$; if $f$ is injective and $U,V\subset X$, then $f(U\cap V)=f(U)\cap f(V)$.}
\end{proof}
\begin{lemma} Take a covering space and map $p:\tilde{X}\rightarrow X$ of $X$. For a map $F:Y\times I\rightarrow X$ and a map $\tilde{F}_0:Y\times\{0\}\rightarrow \tilde{X}$ lifting $F|_{Y\times \{0\}}$, there is a unique map $\tilde{F}:Y\times I\rightarrow \tilde{X}$ lifting $F$ and $\tilde{F}|_{Y\times \{0\}}=\tilde{F}_0$.
\end{lemma}
\begin{proof} Take a point $y_0\in Y, t\in I$. Then since $X$ has a covering space, $F(y_0,t)$ has an open neighborhood $U_{t}$ of $F(y_0,t)$ which is evenly covered. Thus, taking the neighborhood $N_t\times (a_t,b_t)\subset F^{-1}(U_t)$ of $(y_0,t)$, we get $F(N_t\times (a_t,b_t))\in U$. Now since $\{N_t\times (a_t,b_t):t\in I\}$ is an open cover of $\{y_0\}\times I$, which is compact set, we may choose a finite subcover, $\{N_i\times (a_i,b_i):i\in \{0,\cdots,m\}\}$, which also gives a finite partition $0=t_0<t_1<\cdots<t_m=1$ such that $\{N\times [t_i,t_{i+1}]:t\in\{0,\cdots,m\}\}$ is an open cover of $\{y_0\}\times I$, and $F(N\times [t_i,t_{i+1}])\subset U_i$, taking $N=\cap_{i=0}^m N_i$.

Now we use induction. First, we already have a lifting $\tilde{F}_0|_N$ of $F|_{N\times \{0\}}$. Now assume that we already have a lifting $\tilde{F}$ on $N\times [0,t_{i}]$. For $F(N\times [t_i,t_{i+1}])\subset U_i$, since $U_i$ is evenly covered there exists $\tilde{U}_i\subset \tilde{X}$ so that $p(\tilde{U}_i)=U_i$ and $\tilde{F}(y_0,t_i)\in \tilde{U}_i$. If $\tilde{F}(N\times \{t_i\})$ is not contained in $\tilde{U}_i$, then we may take smaller open $N'\subset N$ so that $\tilde{F}(N\times \{t_i\})\subset \tilde{U}_i$, which is defined as $N'\times \{t_i\}=N\times \{t_i\}\cap \tilde{F|_{N\times \{t_i\}}}^{-1}(\tilde{U}_i)$. Thus we may think that $\tilde{F}(N\times \{t_i\})\subset \tilde{U}_i$. Now we may define $\tilde{F}$ on $N\times [t_i,t_{i+1}]$ as $p^{-1}|_{U_i}\circ F|_{N\times [t_i,t_{i+1}]}$, which is continuous due to the pasting lemma. Repeating this step finitely many times gives $\tilde{F}:N\times I\rightarrow \tilde{X}$.

For the uniqueness, first we show the uniqueness of the lift if $Y=\{y_0\}$ is a point: suppose that $\tilde{F},\tilde{F}'$ are two lifts of $F:\{y_0\}\times I\rightarrow X$. such that $\tilde{F}(y_0,0)=\tilde{F}'(y_0,0)$. We can do the same procedure above, and so take a finite partition $0=t_0<t_1<\cdots<t_m=1$ so that $F(y_0,[t_i,t_{i+1}])\subset U_i$ for some evenly covered $U_i$. Now again use induction, and consider $\tilde{F}|_{\{y_0\}\times [0,t_i]}=\tilde{F}'|_{\{y_0\}\times [0,t_i]}$. Since $[t_i,t_{i+1}]$ is connected, $\tilde{F}(y_0,[t_i,t_{i+1}]$ is connected, and thus must be connected in one of the disjoint open sets $\tilde{U}_i$ satisfying $p(\tilde{U}_i)$. Since $\tilde{F}(t_i)=\tilde{F}'(t_i)$, $\tilde{F}'([t_i,t_{i+1}])\subset \tilde{U}_i$. Since $p$ is injective on $\tilde{U}_i$ and $p\circ \tilde{F}=p\circ \tilde{F}'=F$, $\tilde{F}=\tilde{F}'$ on $[t_i,t_{i+1}]$, which shows that $\tilde{F}=\tilde{F}'$ by induction.

Finally, if $N\times I$ and $M\times I$ overlaps, then since the lifting on $\{y_0\}\times I$ is unique, the lifting on $N\times I\cap M\times I$ is uniquely determined. Thus, using all the neighbors of $y\in Y$, we get the lifting $\tilde{F}:Y\times I\rightarrow \tilde{X}$. This is continuous since this is continuous on each $N\times I$, and this is unique since it is unique on each $\{y_0\}\times I$.
\end{proof}

\noindent\rule{\textwidth}{1pt}
\newline