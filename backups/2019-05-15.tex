\mytitle{Algebraic Topology}
\begin{exmp}[The shrinking wedge of circles and countable wedge sum of circles.] Consider the space $X=\cup_{n=1}^{\infty} C_n\subset \mathbb{R}^2$, where $C_n$ is a circle with radius $1/n$ and center $(1/n,0)$. Now consider the retractions $r_n:X\rightarrow C_n$ which is defined as $r_n|_{C_n}=id_{C_n}$ and $r_n|_{X-C_n}=(0,0)$. Each retraction induces a homomorphism $\rho_n:\pi_1(X)\rightarrow \pi_1(C_n)\simeq \mathbb{Z}$, which is surjective since $r_n$ is a retraction. Now we define the direct \textit{product} of these maps and define $\rho:\pi_1(X)\rightarrow \prod_{n=1}^\infty \mathbb{Z}$. Choose a sequence of integers $\{k_n\}$. Define a map $f:I\rightarrow X$ which is a glued map of $f_n:[1-1/n,1-1/(n+1)]\rightarrow C_n$, where $f_n$ wraps $C_n$ $k_n$ times. Notice that $f(0)=f(1)=(0,0)$ Due to the pasting lemma, this map is continuous on $[0,1)$. At the point $1$, the open neighborhood of $(0,0)$ always contains all but finitely many of the circles $C_n$, thus its inverse is open. Therefore $f$ is a loop on $X$, and by definition $\rho([f])=(k_n)$. This shows that $\rho$ is surjective, hence $\pi_1(X)$ is uncountable.\marginnote{The best representation of $\pi_1(X)$ group needs shape theory, which will not going to be treated here.}

However, the fundamental group of countable wedge sum of circles is the free group generated by $S=\{s_n\}_{n\in \mathbb{Z}}$, which is smaller then the set of words generated by $S'=S\cup \{s_n^{-1}|n\in \mathbb{Z}\}$, i.e. $\cup_{n\in \mathbb{N}}S'^n$. Notice that $S'$ is again infinitely countable, hence we may take a bijection $i:S'\rightarrow \mathbb{N}\cup \{0\}$. Now take a bijection $b:\cup_{n\in \mathbb{N}}S'^n\rightarrow \mathbb{N}$ as $b(s_{i_1}\cdots s_{i_j})=2^{i(s_{i_1})}3^{i(s_{i_2})}\cdots p_j^{i(s_{i_j})}$, where $p_j$ is the $j$-th prime number. Therefore the fundamental group of countable wedge sum of circles is countable, which cannot be equal to $\pi_1(X)$.
\end{exmp}

\begin{prop} Let $X$ be a path connected space and choose $x_0\in X$. Consider we have a collection of 2-cells $e_\alpha^2\simeq D^2-S^1$ and a collection of maps $\phi_\alpha:S^1\rightarrow X$. Define $Y=X\bigcup \cup_{\alpha}\textrm{cl}(e_\alpha^2)/(\phi_\alpha(s)\sim s, \forall s\in \partial e_\alpha^2)$. Choose $s_0\in S^1$ as a basepoint, and take a path $\gamma_\alpha$ in $X$ from $x_0$ to $\phi_\alpha(s_0)$. Take $N$ the normal subgroup of $\pi_1(X,x_0)$ which is generated by all the loops $\gamma_\alpha \phi_\alpha \bar{\gamma_\alpha}$.
\begin{enumerate}[label=(\alph*)]
\item $\pi_1(Y)\simeq \pi_1(X)/N$.
\item If we attach $n$-cells with $n>2$ rather then 2-cells, then $\pi_1(Y)\simeq \pi_1(X)$.
\item If $X$ is a cell complex and $X^2$ is the 2-skeleton of $X$, then $\pi_1(X^2)\simeq \pi_1(X)$.
\end{enumerate}
\end{prop}
\begin{proof}
\begin{enumerate}[label=(\alph*)]
\item We define $Z$ as a expansion of $Y$, which is, attaching regular strips $I\times I$ on each $\gamma_\alpha$ following $I\times \{0\}$, and attaching $\{1\}\times I$ line on $\textrm{cl}(e_\alpha^2)$. Reducing the height of the strip, it is possible to deformation retract $Z$ to $Y$. Also, choose $y_\alpha$ from $e_\alpha^2$ which is not on the attached part of the strip.

Now take $A=Z-\cup_\alpha \{y_\alpha\}$ and $B=Z-X$. Then since $e_\alpha^2-y_\alpha$ can be deformation retract to its boundary, $A$ deformation retracts to $X$, and $B$ is contractible. Furthermore, the intersection can be deformation retracted into the wedge sum of $S^1$'s, whose fundamental group is $*_\alpha \mathbb{Z}$ and generated by $[\gamma_\alpha \phi_\alpha \bar{\gamma_\alpha}]$'s. This is trivial in $B$, therefore $\pi_1(Y)\simeq \pi_1(X)/N$ where $N$ is generated by $\gamma_\alpha \phi_\alpha \bar{\gamma_\alpha}$'s.
\item All the argument is same except the intersection is a wedge sum of $S^{n-1}$'s, where $n>2$, hence the intersection is contractible. Therefore $\pi_1(Y)\simeq \pi_1(X)$.
\item Let $f:I\rightarrow X$ be a loop at the basepoint $x_0\in X^2$. Since the image is compact, it is in $X^n$ for some finite $n$.\marginnote{\begin{prop} A compact subspace in CW complex is contained in a finite subcomplex.
\end{prop}
\begin{proof} Suppose that a compact subspace $C$ of CW complex $X$ meets infinitely many cells in $X$, therefore we can choose an infinite set $S=\{x_1,x_2,\cdots\}\subset C$ where all $x_i$ lies in different cells. Notice that $S\cap X^0$ is closed in $X^0$, since it is the set of discrete points. Now suppose that $S\cap X^{n-1}$ is closed in $X^{n-1}$. For each $e_\alpha^n$ in $X$, $\phi_\alpha^{-1}(S)$ is closed in $\partial D_\alpha^n$ for attaching map $\phi_\alpha$, and since there is at most one point of $\Phi_\alpha^{-1}(S)$ in $\textrm{cl}(e_\alpha^n)$ where $\Phi_\alpha$ is a characteristic map, $\Phi_\alpha^{-1}(S)$ is closed in $\textrm{cl}(e_\alpha^n)$. Thus $S$ is closed in $X$. Using same argument shows that any subspace of $S$ is closed, hence $S$ has discrete topology. Since $C$ is compact, $S$ is finite, contradiction. Thus $C$ intersects with finitely many cells. Furthermore, since the closure of cell is compact, closure of cells also intersects with finitely many cells.

Now if we show that every cells are contained in finite subcomplex, then we can show that $C$ is contained in finite subcomplex. Indeed $e_\alpha^1$ is in finite subcomplex which is line, and the boundary of $e_\alpha^n$ is in $X^{n-1}$ and compact thus in finite subcomplex $A\subset X^{n-1}$, therefore $e_\alpha^n$ is in finite subcomplex $A\cup e_\alpha^n$.
\end{proof}} By (b), $f$ is homotopic to a loop in $X^2$, hence $\pi_1(X^2)\rightarrow \pi_1(X)$ is surjective. Now choose $f:I\rightarrow X^2$ a loop which is nullhomotopic in $X$ by a homotopy $F:I\times I\rightarrow X$. Since the image is compact, it is in $X^n$ for some finite $n>2$. Since $\pi_1(X^2)\rightarrow \pi_1(X^n)$ is bijective by (b), and $f$ is nullhomotopic in $X^n$, $f$ is nullhomotopic in $X^2$. 
\end{enumerate}
\end{proof}
\noindent\rule{\textwidth}{1pt}
\newline