\mytitle{Algebraic Topology}
\begin{exmp} Orientable surface with $g$ genus, $M_g$, has a cell structure with one 0-cell, $2g$ 1-cells, and one 2-cell. Before attaching 2-cell, we have the wedge sum of $2g$ cells, which gives the fundamental group $\langle a_1,b_1,\cdots, a_g, b_g\rangle$, i.e. a free group with $2g$ generators. Attaching 2-cell, the boundary of 2-cell is represented by the product of commutators of the generators, $[a_1,b_1][a_2,b_2]\cdots[a_g,b_g]$.\marginnote{For $a,b\in G$, $[a,b]=aba^{-1}b^{-1}$.} Thus, $\pi_1(M_g)\simeq \langle a_1,b_1,\cdots,a_g,b_g|[a_1,b_1]\cdots[a_g,b_g]\rangle$.
\end{exmp}
\begin{cor} If $g\neq h$, $M_g\not\simeq M_h$.
\end{cor}
\begin{proof}
The abelianization of $\pi_1(M_g)$ is the product of $2g$ copies of $\mathbb{Z}$, so if $M_g\simeq M_h$ then $g=h$.
\end{proof}

\begin{cor} For every group $G$ there is a 2-dimensional cell complex $X_G$ with $\pi_1(X_G)\simeq G$.
\end{cor}
\begin{proof} Since every group is a quotient of free group,\marginnote{For $G$, take a free group generated by all the elements of $G$. Now put all the multiplication relations as the relation condition of free group, and take the quotient. We can write the result as $\langle G|g_1g_2g_3^{-1}=e,\forall g_1,g_2\in G, g_1g_2=g_3\rangle$.} choose a representation $G=\langle g_\alpha|r_\beta\rangle$. Now attach 2-cells $e_\beta^2$ to $\vee_\alpha S_\alpha^1$ by the loops specified by the relations $r_\beta$.
\end{proof}

\begin{exmp} Take $G=\langle a|a^n\rangle$. Then $X_G$ is a circle $S^1$ with a cell $e^2$ attached by the map $z\mapsto z^n$. 
\end{exmp}

\begin{prop} For a space $X$, take a covering space $\tilde{X}$ and covering map $p:(\tilde{X},\tilde{x}_0)\rightarrow (X,x_0)$. Then the induced homomorphism $p_*:\pi_1(\tilde{X},\tilde{x}_0)\rightarrow \pi_1(X,x_0)$ is injective, and the image subgroup $p_*(\pi_1(\tilde{X},\tilde{x}_0))\leq \pi_1(X,x_0)$ consists of the homotopy classes of loops in $X$ based at $x_0$ whose lifts in $\tilde{X}$ starting at $\tilde{x}_0$ are loops.
\end{prop}
\begin{proof}
Take a loop $\tilde{f}_0:I\rightarrow \tilde{X}$ where $f_0=p\circ \tilde{f}_0$ is path homotopic to trivial loop $f_1$ by loop homotopy $f_t:I\rightarrow X$. By the homotopy lifting property,\marginnote{Recall(homotopy lifting property): Given a covering space $p:\tilde{X}\rightarrow X$, a homotopy $f_t:Y\rightarrow X$, and a map $\tilde{f}_0:Y\rightarrow \tilde{X}$ lifting $f_0$, there is a unique homotopy $\tilde{f}_t:Y\rightarrow \tilde{X}$ which lifts $f_t$.} we have a loop homotopy $\tilde{f}_t:I\rightarrow \tilde{X}$ which is lifting of $f_t$ and homotopy between $\tilde{f}_0$ and $\tilde{f}_1$, but since $f_1$ is a trivial loop, $\tilde{f}_1$ is also a trivial loop, thus $\ker(p_*)=0$ and $p_*$ is injective. Now if $f:I\rightarrow X$ is a loop based on $x_0$ whose lift $\tilde{f}:I\rightarrow \tilde{X}$ starting at $\tilde{x}_0$ is loop, then $p_*([\tilde{f}])=[f]$, thus $[f]\in p_*(\pi_1(\tilde{X},\tilde{x}_0))$. Conversely, if $f:I\rightarrow X$ is a loop based on $x_0$ where there exists a loop $\tilde{f}':I\rightarrow \tilde{X}$ based on $\tilde{x}_0$ with $p_*([\tilde{f}'])=[f]$, then $[p\circ \tilde{f}']=[f]$, therefore there is a lifting loop $\tilde{f}$ of $f$ based on $\tilde{x}_0$ satisfying $p\circ \tilde{f}=f$.
\end{proof}

\begin{prop} The number of sheets of a covering space $p:(\tilde{X},\tilde{x}_0)\rightarrow (X,x_0)$, where $X,\tilde{X}$ are path connected, equals the index of $p_*(\pi_1(\tilde{X},\tilde{x}_0))$ in $\pi_1(X,x_0)$.
\end{prop}
\begin{proof}
Denote $H=p_*(\pi_1(\tilde{X},\tilde{x}_0))$. Define $\Phi:G/H\rightarrow p^{-1}(x_0)$ as $\Phi(H[g])=\tilde{g}(1)$ for the loops $g$ in $X$ based on $x_0$. Since $h\in H$ then $\tilde{h}$ is also a loop, $\widetilde{h\cdot g}=\tilde{h}\cdot \tilde{g}$ has a same endpoint with $\tilde{g}$. Therefore the map is well defined. Since $\tilde{X}$ is path connected, $\tilde{x}_0$ can be joined to any point $y\in p^{-1}(x_0)$ by a path $\tilde{g}$, thus we can define $g=p\circ \tilde{g}$ such that $\Phi(H[g])=y$, therefore $\Phi$ is surjective. Now suppose that $\Phi(H[g_1])=\Phi(H[g_2])$. This implies that $g_1\cdot \bar{g}_2$ lifts to a loop in $\tilde{X}$ based on $\tilde{x}_0$, therefore $[g_1][g_2]^{-1}\in H$, hence $H[g_1]=H[g_2]$ so $\Phi$ is injective.
\end{proof}

\begin{prop}[Lifting criterion] Take a covering space $p:(\tilde{X},\tilde{x}_0)\rightarrow (X,x_0)$ and a continuous map $f:(Y,y_0)\rightarrow (X,x_0)$ where $Y$ is path connected and locally path connected space.\marginnote{A space $X$ is locally path connected if for all $x\in X$ and open neighborhood $U$ of $x$, there is a path connected open neighborhood $V$ of $x$ such that $x\in V\subset U$.} Then a lift $\tilde{f}:(Y,y_0)\rightarrow (\tilde{X},\tilde{x}_0)$ exists if and only if $f_*(\pi_1(Y,y_0))\subset p_*(\pi_1(\tilde{X},\tilde{x}_0))$.
\end{prop}
\begin{proof}
Suppose that the lift exists. Then since $f=p\circ \tilde{f}$, $f_*=p_*\circ \tilde{f}_*$, thus $f_*(\pi_1(Y,y_0))=p_*(\tilde{f}_*(\pi_1(Y,y_0))\subset p_*(\pi_1(\tilde{X},\tilde{x}_0))$. Now suppose that $f_*(\pi_1(Y,y_0))\subset p_*(\pi_1(\tilde{X},\tilde{x}_0))$. Let $y\in Y$ and $\gamma$ be a path from $y_0$ to $y$, which exists due to the path connectivity of $Y$. The path $f\circ \gamma$ in $X$ with starting point $x_0$ has a unique lift $\widetilde{f\circ \gamma}$ starting at $\tilde{x}_0$. Now define $\tilde{f}(y)=\widetilde{f\circ \gamma}(1)$. Notice that $p\circ \tilde{f}(y)=p\circ \widetilde{f\circ \gamma}(1)=f\circ \gamma(1)=f(y)$. Now choose another path $\gamma'$ from $y_0$ to $y$. Since $h_0=(f\circ \gamma')\cdot (\overline{f\circ \gamma})=f\circ(\gamma'\cdot \bar{\gamma})$ is a loop with basepoint $x_0$, $[h_0]\in f_*(\pi_1(Y,y_0))\subset p_*(\pi_1(\tilde{X},\tilde{x}_0))$. This shows that there is a path homotopic loop $h_1$ with $h_0$ by path homotopy $h_t$ lifts to the loop $\tilde{h}_1$ in $\tilde{X}$ with basepoint $\tilde{x}_0$, and by homotopy lifting property, there is a lifting $\tilde{h}_t$. Thus we get a lifted loop $\tilde{h}_0$ of $h_0$. Due to the uniqueness of lifted paths, $\tilde{h}_0=(\widetilde{f\circ \gamma'})\cdot(\overline{\widetilde{f\circ \gamma}})$. thus $\tilde{f\circ \gamma}(1)=\tilde{f\circ \gamma'}(1)$, and so $\tilde{f}$ is well defined.

Now let $U\subset X$ be an open neighborhood of $f(y)$ with a lift $\tilde{U}\subset \tilde{X}$ containing $\tilde{f}(y)$ such that $p:\tilde{U}\rightarrow U$ is a homeomorphism. Since $Y$ is locally path connected, we may choose a path connected open neighborhood $V\subset f^{-1}(U)$ of $y$ with $f(V)\subset U$. Now take a path $\gamma$ from $y_0$ to $y$ and a path $\eta$ from $y$ to $y'\in V$. Then a path $(f\circ \gamma)\cdot (f\circ \eta)$ in $X$ has a lift $(\widetilde{f\circ \gamma})\cdot (\widetilde{f\circ \eta})$, where $\widetilde{f\circ \eta}=p|_{\tilde{U}}^{-1}\circ f\circ \eta$. Since the endpoint of the last path is in $\tilde{U}$, $\tilde{f}(y')\in \tilde{U}$, thus $\tilde{f}(V)\subset \tilde{U}$. Furthermore, $\tilde{f}(y')=\widetilde{f\circ \eta}(1)=p|_{\tilde{U}}^{-1}\circ f\circ \eta(1)=p|_{\tilde{U}}^{-1}\circ f(y')$, $\tilde{f}|_V=p|_{\tilde{U}}^{-1}\circ f$. Since $f$ and $p|_{\tilde{U}}^{-1}$ is continuous, $\tilde{f}$ is continuous on $V$, hence $\tilde{f}$ is continuous.
\end{proof}

\begin{exmp} The locally path connected condition is crucial. Consider the \textbf{extended topologist's sine curve}, defined as $S=\{(x,y):y=\sin\left(\frac{\pi}{x}\right),x\in (0,1]\}\cup \left(\{0\}\times [-1,1]\right)\cup P$, where $P$ is the path connecting $(0,0)$ and $(1,0)$ which does not intersects with previous parts except the endpoints, for example, $P=\{(x-1)^2+(y+1)^2=1:x\in [1,2]\}\cup [0,1]\times \{-2\}\cup \{x^2+(y+1)^2=1:x\in [-1,0]\}$. This space is not locally path connected, since every open neighborhood of $(0,0)$ with radius less then $1$ is not path connected. Consider the map $p:\mathbb{R}\rightarrow S^1$ as $p(\theta)=(\cos\theta,\sin\theta)$ and a continuous map $f:S\rightarrow S^1$, which is defined as the composition of two maps, $s\circ q=f$, where $q:S\rightarrow \overline{S}$ is defined as
\begin{equation}
q(x,y)=\begin{cases}
(x,y),&(x,y)\in P\\
(x,0),&(x,y)\in S-P
\end{cases}
\end{equation}
and $s:\overline{S}\rightarrow S^1$ is defined by mapping the upper line, rightmost half circle, lower line, and leftmost half circle to first, second, third, fourth quadrant of $S^1$ respectively. Notice that $f_*(\pi_1(S))=0$ thus $f_*(\pi_1(S))\subset p_*(\pi_1(\mathbb{R}))$.

Now write $\{0\}\times [-1,1]=L$. WLOG we may assume that $f(L)=1$, by rotating $S^1$ if needed. Now suppose $\tilde{f}:S\rightarrow \mathbb{R}$ is a lift of $f$, i.e. $p\circ \tilde{f}=f$. Since $S-L$ is connected, $\tilde{f}(S-L)$ is connected in $\mathbb{R}$. Also since $p^{-1}\circ f(S-L)=\mathbb{R}-2\pi \mathbb{Z}$, $\tilde{f}(S-L)$ must be included in the interval, which can be chosen as $(0,2\pi)$, WLOG. Since $f$ is surjective, $\tilde{f}(S-L)=(0,2\pi)$. Since $S$ is compact, $[0,2\pi]\subset \tilde{f}(S)$, thus $\{0,2\pi\}\subset \tilde{f}(L)$, and since $f(L)=1=p\circ \tilde{f}(L)$, $\{0,2\pi\}=\tilde{f}(L)$. But since $L$ is connected set, it is contradiction.
\end{exmp}

\begin{prop}[Unique lifting property] For a covering space $p:\tilde{X}\rightarrow X$ and a map $f:Y\rightarrow X$, if two lifts $\tilde{f}_1,\tilde{f}_2:Y\rightarrow \tilde{X}$ of $f$ agree at one point of $Y$ and $Y$ is connected, then $\tilde{f}_1=\tilde{f}_2$.
\end{prop}
\begin{proof}
For $y\in Y$, let $U$ is an evenly covered open neighborhood of $f(y)$, i.e. $p^{-1}(U)$ is a disjoint union of open sets $\tilde{U}_\alpha$ which is homeomorphic to $U$ by the inverse of $p$. Let $\tilde{f}_1(y)\in \tilde{U}_1$ and $\tilde{f}_2(y)\in\tilde{U}_2$. Since $\tilde{f}_1, \tilde{f}_2$ are continuous, we have an open neighborhood $N$ of $y$ such that $\tilde{f}_1(N)\subset \tilde{U}_1,\tilde{f}_2(N)\subset \tilde{U}_2$. If $\tilde{f}_1(y)=\tilde{f}_2(y)$ then $\tilde{U}_1$ and $\tilde{U}_2$ intersects, hence $\tilde{U}_1=\tilde{U}_2=\tilde{U}$, so $p|_{\tilde{U}}\circ \tilde{f}_1|_N=p|_{\tilde{U}}\circ \tilde{f}_2|_N$. Since $p|_{\tilde{U}}$ is homeomorphism, $\tilde{f}_1|_N=\tilde{f}_2|_N$. If $\tilde{f}_1(y)\neq \tilde{f}_2(y)$ then $\tilde{U}_1\neq \tilde{U}_2$, therefore $\tilde{f}_1|_N\neq \tilde{f}_2|_N$. Now we can divide $Y$ into two disjoint sets, $A=\{y\in Y:\tilde{f}_1(y)=\tilde{f}_2(y)\}$ and $B=\{y\in Y:\tilde{f}_1(y)\neq \tilde{f}_2(y)\}$. From the argument right before, $A,B$ are open. Since $Y$ is connected, $A$ or $B$ is empty. Since $A$ is not empty, $B$ is empty, and so for all $y\in Y$, $\tilde{f}_1(y)=\tilde{f}_2(y)$.
\end{proof}
\noindent\rule{\textwidth}{1pt}
\newline