\mytitle{Algebraic Topology}
\begin{defn}\marginnote{This definition implies that every pair of maps $X\times \{0\}\rightarrow Y$ and $A\times I\rightarrow Y$ which agrees on $A\times\{0\}$ can be extended to a map $X\times I\rightarrow Y$.} For the spaces $A\subset X$, if for every map $f_0:X\rightarrow Y$ and for every homotopy $f_t^A:A\rightarrow Y$ with $f_0|_A=f_0^A$, there is a homotopy $f_t:X\rightarrow Y$ with $f_t|_A=f_t^A$, then we call $(X,A)$ has the \textbf{homotopy extension property}.
\end{defn}
\begin{prop} If $A$ is a closed subspace of $X$, then a pair $(X,A)$ has the homotopy extension property if and only if $X\times \{0\}\cup A\times I$ is a retract of $X\times I$.\marginnote{This is true for all space $A$, but the proof is quite long. I'll put the proof of it if I have enough time.}
\end{prop}
\begin{proof} Suppose that $(X,A)$ has the homotopy extension property. Then for an identity map $X\times \{0\}\cup A\times \{0\}\rightarrow X\times \{0\}\cup A\times \{0\}$, we have an extension $X\times I\rightarrow X\times \{0\}\cup A\times \{0\}$, which gives the retraction. For the inverse, take two maps $X\times \{0\}\rightarrow Y$ and $A\times I\rightarrow Y$ which agrees on $A\times \{0\}$. Since $X\times \{0\}$ and $A\times I$ are both closed, we can use \textit{pasting lemma},\marginnote{\textbf{Pasting lemma} says that if $X,Y\subset A$ are \textit{both closed or both open} with $X\cup Y=A$, and if $f:A\rightarrow B$ is a function where $f|_X, f|_Y$ are continuous, then $f$ is continuous. \textit{Proof.} If $U\subset B$ is closed, then $f^{-1}(U)\cap X$ and $f^{-1}(U)\cap Y$ are closed, so their union $f^{-1}(U)$ is closed. Same for open case. \textit{Counterexample.} Take $f:(-\infty,0]\cup \{1\}\rightarrow 1$ with $f(x)=1$ and $g:(0,\infty)\rightarrow \mathbb{R}$ with $g(x)=x$.} so that $X\times \{0\}\cup A\times I\rightarrow Y$ is continuous. Using the retraction $X\times I\rightarrow X\times\{0\}\cup A\times I$, we get the extension $X\times I\rightarrow Y$.
\end{proof}
\begin{exmp} Take $(I,A)$ with $A=\{0\}\cup\bigcup_{n=1}^{\infty}\{\frac{1}{n}\}$. Since there is no retraction $I\times I \rightarrow I\times \{0\} \cup A\times I$, $(I,A)$ does not have the homotopy extension property.
\end{exmp}
\begin{prop} A pair $(X,A)$ has the homotopy extension property if $A$ has a \textbf{mapping cylinder neigborhood} in $X$, which means, there is a closed boundary $N$ of $A$, which gives $N-\partial N$ as an open boundary of $A$, such that there is a map $f:\partial N\rightarrow A$ and a homeomorphism $h:M_f\rightarrow N$ with $h|_{A\cup \partial N}=1_{A\cup \partial N}$.
\end{prop}
\begin{proof} Since $I\times I$ retracts on $I\times \{0\}\cup \partial I\times I$, $\partial N\times I\times I$ retracts on $\partial N\times I\times\{0\}\cup \partial N\times \partial I\times I$. This retraction induces a retraction of $M_f\times I$ onto $M_f\times \{0\}\cup(A\cup \partial N)\times I$, hence $(M_f,A\cup \partial N)$ has the homotopy extension property. Since $M_f\simeq N$, $(N,A\cup \partial N)$ also has the homotopy extension property. Now for any map $f_0:X\rightarrow Y$ and a homotopy $f_t^A:A\rightarrow Y$ with $f_0|_A = f_0^A$, take the constant homotopy $f_t^{X-(N-\partial N)}:X-(N-\partial N)\rightarrow Y$ which is same with $f_0|_{X-(N-\partial N)}$. By using these, we now have the homotopy $f_t^{A\cup \partial N}:A\cup \partial N\rightarrow Y$. By the homotopy extension property of $(N,A\cup \partial N)$, we get the extension homotopy $f_t^{N}:N\rightarrow Y$, which agrees with $f_t^{X-(N-\partial N)}$ on $(N-\partial N)\times N$. This is closed set, so by pasting lemma, we get the total homotopy.
\end{proof}
\begin{prop} If $(X,A)$ is a CW pair, then $X\times \{0\}\cup A\times I$ is a deformation retract of $X\times I$, hence $(X,A)$ has the homotopy extension property.
\end{prop}
\begin{proof} There is a deformation retraction $r:D^n\times I \rightarrow D^n\times {0}\cup \partial D^n\times I$, defined by the projection from $(0,2)\in D^n\times \mathbb{R}$ for example. Thus there is a deformation retraction of $X^n\times I$ onto $X^n\times \{0\}\cup (X^{n-1}\cup A^n)\times I$. Now we deformation retract $X^n\times I$ onto $X^{n}\times \{0\}\cup (X^{n-1}\cup A^n)\times I$ during the $t$-interval $[1/2^{n+1},1/2^n]$, then the infinite concatenation of a homotopies is a deforamtion retraction of $X\times I$ onto $X\times\{0\}\cup A\times I$: at $t=0$ it is continuous on $X^n\times I$, and since $X$ has weak topology, the given map is continuous.
\end{proof}
\noindent\rule{\textwidth}{1pt}
\newline