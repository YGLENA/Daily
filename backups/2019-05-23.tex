\mytitle{Algebraic Topology}
\begin{defn} For two covering spaces $p_1:\tilde{X}_1\rightarrow X, p_2:\tilde{X}_2\rightarrow X$, the \textbf{isomorphism} between these covering spaces is a homeomorphism $f:\tilde{X}_1\rightarrow \tilde{X}_2$ satisfying $p_1=p_2\circ f$.
\end{defn}
\begin{cor} If $f:\tilde{X}_1\rightarrow \tilde{X}_2$ is an isomorphism between two covering spaces $p_1:\tilde{X}_1\rightarrow X, p_2:\tilde{X}_2\rightarrow X$, then $f^{-1}$ is also.
\end{cor}
\begin{proof}
Since $f$ is homeomorphism, $f^{-1}:\tilde{X}_2\rightarrow \tilde{X}_1$ is also homeomorphism, and since $p_1=p_2\circ f$, $p_2=p_1\circ f^{-1}$.
\end{proof}

\begin{prop} If $X$ is path connected and locally path connected, then two path connected covering spaces $p_1:\tilde{X}_1\rightarrow X$ and $p_2:\tilde{X}_2\rightarrow X$ are isomorphic by an isomorphism $f:\tilde{X}_1\rightarrow \tilde{X}_2$ taking a basepoint $\tilde{x}_1\in p_1^{-1}(x_0)$ to $\tilde{x}_2\in p_2^{-1}(x_0)$ if and only if $p_{1*}(\pi_1(\tilde{X}_1,\tilde{x}_1))=p_{2*}(\pi_1(\tilde{X}_2,\tilde{x}_2))$.
\end{prop}
\begin{proof} If we have the isomorphism, then since $p_1=p_2\circ f$ we have $p_{1*}(\pi_1(\tilde{X}_1,\tilde{x}_1))=p_{2*}\circ f_*(\pi_1(\tilde{X}_1,\tilde{x}_1))\subset p_{2*}(\pi_1(\tilde{X}_2,\tilde{x}_2))$ and since $p_2=p_1\circ f^{-1}$ we have $p_2{*}(\pi_1(\tilde{X}_2,\tilde{x}_2))\subset p_{1*}(\pi_1(\tilde{X}_1,\tilde{x}_1))$, thus we get the desired result. Conversely suppose that $p_{1*}(\pi_1(\tilde{X}_1,\tilde{x}_1))=p_{2*}(\pi_1(\tilde{X}_2,\tilde{x}_2))$. Since $\tilde{X}_1$ is path connected and locally path connected,\marginnote{\begin{lemma} If $p:\tilde{X}\rightarrow X$ is a covering map and $X$ is locally path connected, then $\tilde{X}$ is locally path connected.\end{lemma}\begin{proof} Take $\tilde{x}\in \tilde{X}$ and its open neighborhood $\tilde{x}\in V$. Let $U$ be an evenly connected open neighborhood of $x=p(\tilde{x})$. Denote the evenly covering open set of $p^{-1}(U)$ containing $\tilde{x}$ as $U_\alpha$. Since $U_\alpha$ is homeomorphic to $U$, $V\cap U_\alpha$ is homeomorphic to $p(V\cap U_\alpha)=p(V)\cap U$, which is an open set containing $x$. Since $X$ is locally path connected, we may take an path connected open neighborhood $W$ of $x$ included in $p(V)\cap U$. Taking inverse image, $p|_{U_\alpha}^{-1}(W)$, gives a path connected open neighborhood of $\tilde{x}$ in $V$.  
\end{proof}} and by the given condition of fundamental groups, we can use the lifting criterion and take the lifting of $p_1$ to $\tilde{p}_1:(\tilde{X}_1,\tilde{x}_1)\rightarrow (\tilde{X}_2,\tilde{x}_2)$ with $p_2\circ \tilde{p}_1=p_1$. Similarly we can take the lifting of $p_2$ to $\tilde{p}_2:(\tilde{X}_2,\tilde{x}_2)\rightarrow (\tilde{X}_1,\tilde{x}_1)$ with $p_1\circ \tilde{p}_2=p_2$. Notice that $p_2\circ (\tilde{p}_1\circ \tilde{p}_2)=p_2$, thus $\tilde{p}_1\circ \tilde{p}_2$ is the lift of $p_2$ by $p_2$. Since $1_{\tilde{X}_2}$ is also the lift of $p_2$ by $p_2$, by the unique lifting property, $\tilde{p}_1\circ \tilde{p}_2=1_{\tilde{X}_2}$. Similarly $\tilde{p}_2\circ \tilde{p}_1=1_{\tilde{X}_1}$, therefore $\tilde{p}_1$ and $\tilde{p}_2$ are inverse isomorphisms.
\end{proof}

\begin{thm}[Covering space classification theorem.] Let $X$ be a path connected, locally path connected, and semilocally simply connected space. Then there is a bijection between the set of basepoint-preserving isomorphism classes of path connected covering spaces $p:(\tilde{X},\tilde{x}_0)\rightarrow (X,x_0)$ and the set of subgroups of $\pi_1(X,x_0)$, which is obtained by associating the subgroup $p_*(\tilde{X},\tilde{x}_0)$ to the covering space $(\tilde{X},\tilde{x}_0)$. If basepoints are ignored, this correspondence gives a bijection between isomorphism classes of path connected covering spaces $p:\tilde{X}\rightarrow X$ and conjugacy classes of subgroups of $\pi_1(X,x_0)$.
\end{thm}
\begin{proof}
The first statement is proven by the previous propositions. For the second statement, we prove that for a covering space $p:(\tilde{X},\tilde{x}_0)\rightarrow (X,x_0)$, if we change the basepoint $\tilde{x}_0$ to some point of $p^{-1}(x_0)$, then this corresponds to taking the conjugate of $p_*(\pi_1(\tilde{X},\tilde{x}_0))$ in $\pi_1(X,x_0)$. Choose $\tilde{x}_1\in p^{-1}(x_0)$ and take $\tilde{\gamma}$ a path from $\tilde{x}_0$ to $\tilde{x}_1$. Then $\gamma=p\circ \tilde{\gamma}$ is a loop in $X$ with basepoint $x_0$. Define $H_i=p_*(\pi_1(\tilde{X},\tilde{x}_i))$ for $i=0,1$. For a loop $\tilde{f}:I\rightarrow \tilde{X}$ with basepoint $\tilde{x}_0$, $\bar{\tilde{\gamma}}\cdot \tilde{f}\cdot \tilde{\gamma}$ is a loop at $\tilde{x}_1$. Thus $[\gamma]^{-1}H_0[\gamma]\subset H_1$. Similarly we can show $[\gamma]H_1[\gamma]^{-1}\subset H_0$, thus $[\gamma]^{-1}H_0[\gamma]=H_1$. Conversely, take $H_0= p_*(\pi_1(\tilde{X},\tilde{x}_0))$ and let $H_1=g^{-1}H_0 g$ be a conjugate subgroup. Choose a loop $\gamma$ representing $g$, and lift it to $\tilde{\gamma}$ starting at $\tilde{x}_0$. Now, take a loop $\tilde{f}:I\rightarrow \tilde{X}$ with basepoint $\tilde{x}_0$, then $g^{-1}p_*([\tilde{f}])g=[\gamma]^{-1}p_*([f])[\gamma]=p_*([\tilde{\gamma}]^{-1}[\tilde{f}][\tilde{\gamma}])=p_*([\bar{\tilde{\gamma}}\cdot \tilde{f}\cdot \tilde{\gamma}])\in p_*(\pi_1(\tilde{X},\tilde{x}_1))$, thus $H_1\subset p_*(\pi_1(\tilde{X},\tilde{x}_1))$. Similarly, using $H_0=gH_1g^{-1}$, we get $p_*(\pi_1(\tilde{X},\tilde{x}_1))\subset H_1$, thus $H_1=p_*(\pi_1(\tilde{X},\tilde{x}_1))$. Therefore the conjugation gives the basepoint change.
\end{proof}

\begin{defn} A simply connected cover of $X$, if exists, is called the \textbf{universal cover}.\marginnote{Due to the covering space classification theorem, the universal cover is the covering space of every path connected covering space. Since it is unique up to isomorphism, it is called \textit{the} universal cover.}
\end{defn}

\begin{defn} Let $p:\tilde{X}\rightarrow X$ be a covering map and $x_0\in X$. For a loop $\gamma:I\rightarrow X$ with basepoint $x_0$, the \textbf{(right) action of $\pi_1(X,x_0)$ on the fiber $p^{-1}(x_0)$} can be defined as, for $\tilde{x}_1\in p^{-1}(x_0)$, $\tilde{x}_1\cdot [\gamma]=\tilde{\gamma}(1)$, where $\tilde{\gamma}$ is the unique lift of $\gamma$ starting at $\tilde{x}_1$.\marginnote{\begin{defn}For a group $G$ and a set $X$, a \textbf{left group action} is a function $\psi:G\times X\rightarrow X$, where $\psi(g,x)$ is often written as $g\cdot x$, which satisfies (1) $e\cdot x=x$ for all $x\in X$ where $e$ is an identity of $G$, and (2) $(gh)\cdot x=g\cdot (h\cdot x)$ for all $g,h\in G$ and $x\in X$. a \textbf{right group action} is a function $\psi:X\times G\rightarrow X$, where $\psi(x,g)$ is often written as $x\cdot g$, which satisfies (1) $x\cdot e=x$ for all $x\in X$ where $e$ is an identity of $G$, and (2) $x\cdot (gh)=(x\cdot g)\cdot h$ for all $g,h\in G$ and $x\in X$. If a set $X$ is a topological space, then $\phi$ is a \textbf{group action on a space} if $\phi_g:X\rightarrow X$ defined as $\phi_g(x)=\phi(g,x)$ is a homeomorphism.\end{defn}}
\end{defn}

\begin{prop} For a covering map $p:\tilde{X}\rightarrow X$ and $x_0\in X$, the action of $\pi_1(X,x_0)$ on the fiber $p^{-1}(x_0)$ is actually an action.
\end{prop}
\begin{proof}
First we need to show that the action is well defined. Take $\gamma_1,\gamma_2:I\rightarrow X$ with basepoint $x_0$, and $[\gamma_1]=[\gamma_2]$. Then since $\gamma_1$ and $\gamma_2$ are path homotopic, there is a path homotopy $F:I\times I\rightarrow X$ between $\gamma_1$ and $\gamma_2$. Also, there is a lifting of those paths and path homotopy, $\tilde{\gamma}_1, \tilde{\gamma}_2,\tilde{F}$, where $\tilde{\gamma}_{1,2}$ are the lifting of $\gamma_{1,2}$ and $\tilde{F}$ is a lifted path homotopy between $\tilde{\gamma}_1$ and $\tilde{\gamma}_2$. Therefore $\tilde{\gamma}_1(1)=\tilde{\gamma}_2(1)$.

Next, take a constant loop $c_{x_0}$. The lifting of this loop starting at $\tilde{x}_1$ is $c_{\tilde{x}_1}$, therefore $\tilde{x}_1\cdot [c_{x_0}]=\tilde{x}_1$. Finally, take two loops $\gamma_1, \gamma_2$ with basepoint $x_0$. Then there is a lifting $\tilde{\gamma}_1$ of $\gamma_1$ starting at $\tilde{x}_1$, and a lifting $\tilde{\gamma}_2$ of $\gamma_2$ starting at $\tilde{\gamma}_2(1)$. Furthermore, notice that $\tilde{\gamma}_1\cdot \tilde{\gamma}_2$ is a lifting of $\gamma_1\cdot \gamma_2$. Therefore, $(\tilde{x}_1\cdot [\gamma_1])\cdot [\gamma_2]=\tilde{\gamma}(1)\cdot [\gamma_2]=\tilde{\gamma}_2(1)$ and $\tilde{x}_1\cdot([\gamma_1][\gamma_2])=\tilde{x}_1[\gamma_1\cdot \gamma_2]=(\tilde{\gamma}_1\cdot \tilde{\gamma}_2)(1)=\tilde{\gamma}_2(1).$ Thus this is an action.
\end{proof}
\noindent\rule{\textwidth}{1pt}
\newline