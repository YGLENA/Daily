\mytitle{Algebraic Topology}
\begin{exmp} Since every paths of convex subset $X\subset \mathbb{R}^n$ with same endpoints are path homotopic, $\pi_1(X,x_0)=0$.
\end{exmp}

\begin{defn} For $x_0,x_1\in X$, suppose that there is a path $h:I\rightarrow X$ whose endpoints are $x_0,x_1$. Then a \textbf{change-of-basepoint map} $\beta_h:\pi_1(X,x_1)\rightarrow \pi_1(X,x_0)$ is defined as\marginnote{This map is well defined because the path product is associative under path homotopy, which is proven in the associativity of $\pi_1(X,x_0)$. Furthermore if $f$ is a loop based on $x_1$, then $(h\cdot f)\cdot \bar{h}$ is a loop based on $x_0$.}
\begin{equation}
\beta_h([f])=[h\cdot f\cdot \overline{h}].
\end{equation}
\end{defn}

\begin{prop} For $x_0,x_1\in X$ and a path $h:I\rightarrow X$ whose endpoints are $x_0,x_1$, the change-of-basepoint map $\beta_h:\pi_1(X,x_1)\rightarrow \pi_1(X,x_0)$ is isomorphism. 
\end{prop}
\begin{proof} $\beta_h$ is homomorphism because $\beta_h([f][g])=\beta_h([f\cdot g])=[h\cdot f\cdot g\cdot \overline{h}]=[h\cdot f\cdot \overline{h}\cdot h\cdot g\cdot \overline{h}]=[h\cdot f\cdot \overline{h}][h\cdot g\cdot \overline{h}]=\beta_h([f])\beta_h([g])$. Also $\beta_h$ is isomorphism with inverse $\beta_{\overline{h}}$ since $\beta_h\circ \beta_{\overline{h}}([f])=[h\cdot \overline{h}\cdot f\cdot h\cdot \overline{h}]=[f]$, and exchanging $h,\overline{h}$ gives $\beta_{\overline{h}}\circ\beta_h([f])=[f]$.
\end{proof}

\begin{cor} If $X$ is path connected, then $\pi_1(X,x_0)\simeq \pi_1(X)$, for some group $\pi_1(X)$.
\end{cor}
\begin{proof} Since $X$ is path connected, for any $x_0,x_1\in X$, there is a path connecting $x_0$ and $x_1$, and thus $\pi_1(X,x_0)\simeq \pi_1(X,x_1)$.
\end{proof}

\begin{defn} A space $X$ is \textbf{simply connected} if $X$ is path connected and $\pi_1(X)=0$.
\end{defn}

\begin{prop} A space $X$ is simply connected if and only if there is a unique homotopy class of paths connecting any two points in X. \marginnote{The existence of homotopy class implies the path connectivity of $X$.}
\end{prop}
\begin{proof} If $X$ is simply connected, then for any paths $f,g:I\rightarrow X$ with same endpoints $x_0,x_1$, $f\cdot \overline{g}\simeq c_{x_0}$ and $\overline{g}\cdot g\simeq c_{x_1}$. Therefore, $f\simeq f\cdot \overline{g}\cdot g\simeq g$. Conversely, if there is a unique homotopy class of paths connecting any two points in $X$, then taking the paths connecting $x_0$ to itself gives $\pi_1(X,x_0)=0$.
\end{proof}

\begin{defn} For the spaces $X,Y,A$ and continuous maps $f:A\rightarrow X$ and $p:Y\rightarrow X$, we call a continuous map $\tilde{f}:A\rightarrow Y$ a \textbf{lift} of $f$ if $p\circ \tilde{f}=f$.
\end{defn}

%\begin{exmp} Not every function has its lift. Consider the \textbf{extended topologist's sine curve}, defined as $S=\{(x,y):y=\sin\left(\frac{\pi}{x}\right),x\in (0,1]\}\cup \left(\{0\}\times [-1,1]\right)\cup P$, where $P$ is the path connecting $(0,0)$ and $(1,0)$ which does not intersects with previous parts except the endpoints, for example, $P=\{(x-1)^2+(y+1)^2=1:x\in [1,2]\}\cup [0,1]\times \{-2\}\cup \{x^2+(y+1)^2=1:x\in [-1,0]\}$. Consider the map $p:\mathbb{R}\rightarrow S^1$ as $p(\theta)=(\cos\theta,\sin\theta)$ and a continuous map $f:S\rightarrow S^1$, which is defined as the composition of two maps, $s\circ q=f$, where $q:S\rightarrow \overline{S}$ is defined as
%\begin{equation}
%q(x,y)=\begin{cases}
%(x,y),&(x,y)\in P\\
%(x,0),&(x,y)\in S-P
%\end{cases}
%\end{equation}
%and $s:\overline{S}\rightarrow S^1$ is defined by mapping the upper line, rightmost half circle, lower line, and leftmost half circle to first, second, third, fourth quadrant of $S^1$ respectively.
%
%Now write $\{0\}\times [-1,1]=L$. WLOG we may assume that $f(L)=1$, by rotating $S^1$ if needed. Now suppose $\tilde{f}:S\rightarrow \mathbb{R}$ is a lift of $f$, i.e. $p\circ \tilde{f}=f$. Since $P-L$ is connected, $\tilde{f}(S-L)$ is connected in $\mathbb{R}$. Also since $p^{-1}\circ f(S-L)=\mathbb{R}-2\pi \mathbb{Z}$, $\tilde{f}(S-L)$ must be included in the interval, which can be chosen as $(0,2\pi)$, WLOG. Since $f$ is surjective, $\tilde{f}(S-L)=(0,2\pi)$. Since $S$ is compact, $[0,2\pi]\subset \tilde{f}(S)$, thus $\{0,2\pi\}\subset \tilde{f}(L)$, contradicting $\tilde{f}(L)$
%\end{exmp}
\begin{exmp} Not every function has its lift. Consider the identity map $f:S^1\rightarrow S^1$ and $p:\mathbb{R}\rightarrow S^1$ which is defined as $p(\theta)=(\cos\theta,\sin\theta)$. Then there is no lift $\tilde{f}:S^1\rightarrow \mathbb{R}$ satisfying $p\circ \tilde{f}=f$. Indeed, consider $S^1-x_0$. for $s_0=(1,0)\in S^1$. Since $S^1-s_0$ is connected, $\tilde{f}(S^1-s_0)$ is connected, and since $p^{-1}\circ f(S^1-s_0)=\mathbb{R}-2\pi\mathbb{Z}$, $\tilde{f}(S^1-s_0)$ must be in the interval, which can be taken as $(0,2\pi)$ WLOG. Due to the surjectivity of $f$, $\tilde{f}(S^1-s_0)=(0,2\pi)$ exactly. Now since $S^1$ is compact, $[0,1]\subset \tilde{f}(S^1)$, thus $\{0,1\}\subset \tilde{f}(s_0)$, contradiction.
\end{exmp}

\begin{defn} For a space $X$, a \textbf{covering space} of $X$ is a space $\tilde{X}$ with a map $p:\tilde{X}\rightarrow X$, called a \textbf{covering map}, such that for each $x\in X$, there is an open neighborhood $U\subset X$ of $x$ such that $p^{-1}(U)$ is a union of disjoint open sets, where each are homeomorphic to $U$ by $p$. We call such $U$ \textbf{evenly covered}.
\end{defn}

\begin{exmp} The map $p:\mathbb{R}\rightarrow S^1$ defined as $p(\theta)=(\cos\theta,\sin\theta)$ is a covering map. Thus $\mathbb{R}$ is a covering space of $S^1$.
\end{exmp}

\begin{lemma} For a map $F:Y\times I\rightarrow X$ and a map $\tilde{F}_0:Y\times\{0\}\rightarrow \tilde{X}$ lifting $F|_{Y\times \{0\}}$, there is a unique map $\tilde{F}:Y\times I\rightarrow \tilde{X}$ lifting $F$ and $\tilde{F}|_{Y\times \{0\}}=\tilde{F}_0$.
\end{lemma}
\begin{lemma} For each path $f:I\rightarrow X$ starting at a point $x_0$ and each $\tilde{x}_0\in p^{-1}(x_0)$, there is a unique lift $\tilde{f}:I\rightarrow \tilde{X}$ starting at $\tilde{x}_0$. Also, for each path homotopy $f_t:I\rightarrow X$ starting at $x_0$ and each $\tilde{x}_0\in p^{-1}(x_0)$, there is a unique lifted path homotopy $\tilde{f}_t:I\rightarrow \tilde{X}$ starting at $\tilde{x}_0$.
\end{lemma}
\begin{proof}
For the first statement, take $Y=\{y_0\}$ for some point $y_0$ and use Lemma above. For the second statement, take $F(s,t)=f_t(s)$, then by the first statement we get a unique lift $\tilde{F}_0:I\times \{0\}\rightarrow \tilde{X}$, and by the Lemma above we get a unique lift $\tilde{F}:I\times I\rightarrow \tilde{X}$. Also, $\tilde{F}|_{\{0\}\times I},\tilde{F}|_{\{1\}\times I}$ are the lifts of constant maps $F|_{\{0\}\times I},F|_{\{1\}\times I}$ respectively, hence we may check that for each case the constant map is a lift, and since the uniqueness of lifting, the constant map is the lift. Thus $\tilde{F}$ is path homotopy.
\end{proof}

\begin{thm} $\pi_1(S^1)\simeq \mathbb{Z}$, whose generator is the homotopy class of the loop $\omega(s)=(\cos 2\pi s,\sin 2\pi s)$ based on $(1,0)$.
\end{thm}
\begin{proof}
Let $f:I\rightarrow S^1$ is a loop with basepoint $x_0=(1,0)$. Then by the Lemma above for the path, we have a lifting of the path, $\tilde{f}:I\rightarrow \mathbb{R}$, starting at $0$. Since $p^{-1}(x_0)=\mathbb{Z}\subset \mathbb{R}$, this lifted path ends at $n\in \mathbb{Z}$. Now notice that
\begin{equation}
[\omega]^n=[\underbrace{\omega\cdot\omega\cdot\cdots\cdot\omega}_{n\textrm{ times}}]=[\omega_n]
\end{equation}
where $\omega_n(s)=(\cos 2\pi ns, \sin 2\pi ns)$. Also, the lifting of $\omega_n(s)$ starting at $0$ ends at $n$, by directly checking the path $\tilde{\omega}_n(s)=ns$ is the lifted path. Since $\mathbb{R}$ has a trivial fundamental group, $\tilde{\omega}_n\simeq \tilde{f}$ by some homotopy $H$, and taking $p\circ H$ gives the homotopy between $\omega_n$ and $f$. Therefore $[f]=[\omega_n]$.

To show that the fundamental group of $S^1$ is $\mathbb{Z}$, we need to show that $[\omega_n]=[\omega_m]$ then $n=m$. Choose the homotopy $f_t$ between $f_0=\omega_n$ and $f_1=\omega_m$. By the Lemma above for the homotopy, we have a lifting of the homotopy, $\tilde{f}_t$, whose path starting at $0$, and by the uniqueness of path lifting, $\tilde{f}_0=\tilde{\omega}_n$ and $\tilde{f}_1=\tilde{\omega}_m$. Finally, since $\tilde{f}_t$ is a path homotopy, $\tilde{f}_t(1)$ is constant function, thus $n=\tilde{\omega}_n(1)=\tilde{\omega}_m(1)=m$.
\end{proof}
\noindent\rule{\textwidth}{1pt}
\newline