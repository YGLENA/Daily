\mytitle{Algebraic Topology}
\begin{defn} Let $S$ be the set. The \textbf{free group} $F_S$ \textbf{generated by} $S$ is the free product of groups $*_{s\in S}s_{\mathbb{Z}}$. Here, $s_{\mathbb{Z}}$ is a group with elements $\{s^n|n\in \mathbb{Z}\}$.
\end{defn}

\begin{exmp} An integer group $\mathbb{Z}$ is a free group generated by $\{1\}$.
\end{exmp}

\begin{prop} Let $S$ be the set with size $n$. Then $F_S\simeq \underbrace{\mathbb{Z}*\cdots*\mathbb{Z}}_{n\textrm{ times}}$.
\end{prop}
\begin{proof} This is true since for each objects $s\in S$, $s_{\mathbb{Z}}\simeq \mathbb{Z}$.
\end{proof}

\begin{thm}[The universal property of free product.] Take the collection of groups $G_\alpha$. For any group $H$ and any collection of homomorphisms $\phi_\alpha:G_\alpha\rightarrow H$, there is a unique extension $\phi:*_\alpha G_\alpha\rightarrow H$.
\end{thm}
\begin{proof}
For the existence, we take $\phi(g_1\cdots g_n)=\phi_{\alpha_1}(g_1)\cdots \phi_{\alpha_n}(g_n)$, where $g_i\in G_{\alpha_i}$. If the word $g_1\cdots g_n$ is not reduced, then for each reducing step, i.e. $g_ig_{i+1}$, multiply $\phi_{\alpha_i}(g_i)\phi_{\alpha_{i+1}}(g_{i+1})$. Do the some reducing procedure, and then multiply all the leftings. Due to the associativity of group, the result is always same, independent to the number or sequence of reducing procedure. Therefore this is well defined, and therefore by definition it is homomorphism. Finally, to show the uniqueness, suppose that there is another extension $\psi:*_\alpha G_\alpha\rightarrow H$. Then $\psi(g_1\cdots g_n)=\psi(g_1)\cdots \psi(g_n)$ since $\psi$ is homomorphism. Since $\psi$ is extension, $\psi(g_\alpha)=\phi(g_\alpha)$ for $g_\alpha\in G_{\alpha}$, which gives $\psi(g_1\cdots g_n)=\phi(g_1\cdots g_n)$.
\end{proof}

\begin{thm}[Van Kampen's Theorem.] If $X$ is the union of path connected open sets $A_\alpha$ where each contains the basepoint $x_0\in X$, and if each intersection $A_\alpha\cap A_\beta$ is path-connected, then the homomorphism $\Phi:*_\alpha \pi_1(A_\alpha)\rightarrow \pi_1(X)$ is surjective. Furthermore if $A_\alpha\cap A_\beta \cap A_\gamma$ is path connected, then $\Ker{\Phi}\simeq N$, where $N$ is a group generated by all elements of the form $i_{\alpha\beta}(\omega)i_{\beta\alpha}(\omega)^{-1}$ with $\omega\in \pi_1(A_\alpha\cap A_\beta)$, and thus $\Phi$ induces an isomorphism $\pi_1(X)\simeq *_\alpha \pi_1(A_\alpha)/N$. Here $i_{\alpha\beta}:\pi_1(A_\alpha\cap A_\beta)\rightarrow \pi_1(A_\alpha)$ is the homomorphism induced by the inclusion $A_\alpha\cap A_\beta\hookrightarrow A_\alpha$.
\end{thm}
\begin{proof}
We have already shown the first part: for the given condition, we have shown that all the loop in $X$ at $x_0$ is path homotopic to a product of loops each of which is contained in a single $A_\alpha$. We write those loops as $f_1,\cdots, f_n$ and call $[f_1][f_2]\cdots [f_n]$ a \textbf{factorization} of $[f]$. Thus, for the loops $f_i\subset A_{\alpha_i}$ and $f\in X$, if $i_{\alpha_1}([f_1])\cdots i_{\alpha_n}([f_n])=[f]$, then $[f_1]\cdots [f_n]$ is a factorization of $[f]$. Each factorization is a word in $*_\alpha \pi_1(A_\alpha)$ after reducing it.

Now notice that $i_{\alpha}(i_{\alpha\beta}(\omega))i_{\beta}(i_{\beta\alpha}(\omega)^{-1})=i_{\alpha\cap\beta}(\omega)i_{\alpha\cap\beta}(\omega)^{-1}=e$ where $i_{\alpha\cap\beta}$ is the homomorphism induced by the inclusion $A_\alpha\cap A_\beta\hookrightarrow X$. This implies that $N\leq \Ker \Phi$. Now consider a trivial loop $f$ in $X$, and its factorization $[f_1]\cdots[f_n]$. If we show that $[f_1]\cdots [f_n]\in N$, then $\Ker\Phi\subset N$, and we have proven the theorem.

Since $f_1\cdot \cdots \cdot f_n$ is homotopic to $f$, which is homotopic to constant loop, $c_{x_0}$, we can take a homotopy $F:I\times I\rightarrow X$ from $f_1\cdot \cdots \cdot f_n$ to $c_{x_0}$. Now for each points $F(s,t)$, we have an open neighborhood $U_{s,t}$ which is contained in some $A_{\alpha_s}$. Taking $F^{-1}(U_{s,t})$ gives an open neighborhood $V_{s,t}\subset I\times I$ of $(s,t)$ satisfying $F(V_{s,t})\subset A_{\alpha_{s,t}}$, which makes possible to take the open rectangle $(a_{s,t},b_{s,t})\times (c_{s,t},d_{s,t})$ where $F([a_{s,t}-\epsilon_{s,t},b_{s,t}+\epsilon_{s,t}]\times [c_{s,t}-\epsilon_{s,t},d_{s,t}+\epsilon_{s,t}])\subset A_{\alpha_{s,t}}$, for some $\epsilon_{s,t}>0$. Since $I\times I$ is compact, we may choose a finitely many open rectangles which covers $I\times I$, where its slightly larger closure is fully contained in some $A_\alpha$. Now by choosing all the vertices, and drawing vertical and horizontal lines on them, we can take a partitions $0=s_0<s_1<\cdots <s_m=1$ and $0=t_0<t_1<\cdots<t_n=1$ such that each rectangle $R_{ij}=[s_{i-1},s_{i}]\times [t_{j-1},t_j]$ is mapped by $F$ into a single $A_{ij}$, and a little bit of perturbation of vertical sides of the rectangles $R_{ij}$ so that each point of $I\times I$ lies in at most three $R_{ij}$'s does not changes the result: still $F(R_{ij})\subset A_{ij}$. Number the rectangles from left to right, down to right: $R_1$ for lowest, leftest rectangle, $R_2$ for the right one, and so on.

Notice that $F(0,t)=F(1,t)=x_0$. Now consider $\gamma_r$ be the path separating the first $r$ rectangles, $R_1,\cdots,R_r$, from the remaining rectangles. Then $\gamma_0$ is the bottom edge, where $F|_{\gamma_0}$ is a constant loop, and $\gamma_{mn}$ is the top edge, where $F|_{\gamma_{mn}}$ is $f_1\cdot\cdots\cdot f_n$. Furthermore, all $\gamma_r$ are the loops with basepoint $x_0$.

Now call the corners of the $R_r$'s as vertices. For each vertex $v$, if $F(v)\neq x_0$, then we may choose a path $g_v$ from $x_0$ to $F(v)$ which lies in the intersection of the two or three $A_{ij}$'s corresponding to the $R_r$'s containing $v$, because of the path connectivity of three intersections of $A_\alpha$. This gives a factorization of $[F|_{\gamma_r}]$, by inserting the paths $\bar{g}_v g_v$ where the vertex $v$ exists on $\gamma_r$. Indeed, for the upper edge, we need a bit more trick: choose the path $g_v$ not only included in two $A_\alpha$'s corresponding to the $R_s$'s, but also the one $A_\alpha$, which contains $f_i$, which contains $v$ in its domain. If $v$ is the common endpoint of the domains of two consecutive $f_i$, then $F(v)=x_0$, so we do not need to choose such path.

Now consider the sliding-up of the L-shaped path to ㄱ-shaped path on $I\times I$,
\begin{equation}
\gamma_t(s)=\begin{cases}
(0,1-3st),&s\in [0,\frac{1}{3}]\\
(3s-1,1-t),&s\in [\frac{1}{3},\frac{2}{3}]\\
(1,3(1-t)(1-s)),&s\in [\frac{2}{3},1]
\end{cases}
\end{equation}
By this pushing-up, we may change $\gamma_r$ to $\gamma_{r+1}$ continuously, thus we may change $F|_{\gamma_r}$ to $F|_{\gamma_{r+1}}$ by a homotopy within $A_{r}$. This is just an ordinary loop homotopy, but if we consider the 

Now we are done. The sliding-up process does not changes the representation on fundamental group, since the sliding-up process can be represented as a loop homotopy on each components. The nontrivial change only happens when we change the inclusion of one edge: for example, if the one edge represents a loop $f_i$ contained in both $A_\alpha,A_\beta$, and initially we have $[f_i]_\alpha\in \pi_1(A_\alpha)$, then we need to multiply $i_{\beta\alpha}([f_i]_{\alpha\beta})i_{\alpha\beta}([f_i]_{\alpha\beta})^{-1}$, where $[f_i]_{\alpha\beta}\in\pi_1(A_\alpha\cap A_\beta)$. Thus the whole procedure to changing constant loop to $[f_1]\cdots [f_n]$ is the successive procedure of multiplying above elements between the loops, which gives that $[f_1]\cdots [f_n]\in N$. This proves the theorem.
%Now we call two factorizations $[f_1]\cdots [f_n]$ and $[g_1]\cdots [g_m]$ of $[f]$ are equivalent if by a sequence of following steps or inverses we can change one to another:
%\begin{enumerate}
%\item Change $[f_i][f_{i+1}]$ into $[f_i\cdot f_{i+1}]$ if $[f_i],[f_{i+1}]\in \pi_1(A_\alpha)$,
%\item Change $[f_i]\in \pi_1(A_\alpha)$ as $i_{\beta\alpha}\circ i_{\alpha\beta}^{-1}([f_i])\in \pi_1(A_\beta)$ if $f_i$ is a loop in $A_\alpha\cap A_\beta$.
%\end{enumerate}
%Define $Q=*_\alpha\pi_1(A_\alpha)/N$. Since the first step does not changes the elements of $*_\alpha\pi_1(A_\alpha)$, and the second step changes all the elements of $N$ into identity\marginnote{$i_{\alpha\beta}(\omega)i_{\beta\alpha}(\omega)^{-1}\rightarrow i_{\beta\alpha}\circ i_{\alpha\beta}^{-1}\circ i_{\alpha\beta}(\omega)i_{\beta\alpha}(\omega)^{-1}=i_{\beta\alpha}(\omega)i_{\beta\alpha}(\omega)^{-1}=e$}, equivalent factorizations are same elements of $Q$. If we show that all the factorizations of $[f]$ are equivalent, then we may conclude that the homomorphism $\Phi':Q\rightarrow \pi_1(X)$ is injective, thus $N\simeq \Ker\Phi$ and the theorem is proven.
\end{proof}
\noindent\rule{\textwidth}{1pt}
\newline