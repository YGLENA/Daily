\mytitle{Algebraic topology}
\begin{exmp} Again, $S^2/S^0\simeq S^1\vee S^2$. Indeed, consider $A\subset S^2$ be the arc connecting north and south pole, and define $f,g:A\rightarrow S^1$ as $f(\theta)=(\cos \theta, \sin\theta)$ and $g(\theta)=(1,0)$, where $A$ is parametrized by $\theta$. Since $f,g$ are homotopic by
\begin{equation}
F(\theta,t)=(\cos (t\theta),\sin(t\theta)),
\end{equation}
and $S^2\sqcup_f S^1\simeq S^2/S^0$ and $S^2\sqcup_g S^1\simeq S^1\vee S^2$, we get the desired result.
\end{exmp}

\begin{prop} Suppose $(X,A)$ and $(Y,A)$ satisfies the homotopy extension property, and $f:X\rightarrow Y$ is a homotopy equivalence with $f|_A=1_A$. Then $f$ is a homotopy equivalence $\textrm{ rel }A$.
\end{prop}
\begin{proof}
Let $g:Y\rightarrow X$ be a homotopy inverse of $f$.

First, let $h_t:X\rightarrow X$ be a homotopy from $g\circ f=h_0$ to $1_X=h_1$. Then restricting $h_t$ to $A$, we can say $h_t|_A$ is the homotopy from $g|_A$ to $1_A$, since $f|_A=1_A$. Now using the homotopy extension property of $(Y,A)$, we may construct a homotopy $g_t:Y\rightarrow X$ from $g=g_0$ to $g_1$ where $g_1|_A=1_A$. Now we want to show that $g_1\circ f\simeq 1_X \textrm{ rel } A$.

We have homotopy
\begin{align*}
k_t=\begin{cases}
g_{1-2t}\circ f,&0\leq t\leq \frac{1}{2}\\
h_{2t-1},&\frac{1}{2}\leq t \leq 1
\end{cases}
\end{align*}
between $g_1\circ f$ and $1_X=h_1$, which is not needed to have $k_t|_A=1_A$, but $k_0|_A=k_1|_A=1_A$. Also since $g_t|_A=h_t|_A$, $k_t|_A=k_{1-t}|_A$. Now we define $k_{t,u}:A\rightarrow X$ as
\begin{align*}
k_{t,u}=\begin{cases}
k_{t}|_A, & u\leq 2t-1 \textrm{ or } u\leq -2t+1\\
k_{\frac{u+1}{2}}|_A, & -2t+1\leq u \textrm{ and } -2t+1\leq u
\end{cases}
\end{align*}
Then, along the line $\{0\}\times [0,1]\cup [0,1]\times \{1\}\cup \{1\}\times [0,1]$, we get $k_{t,u}=1_A$. Since $k_{t,u}$ can be thought as the homotopy from $k_{t,0}=k_t|_A:A\times I\rightarrow X$ to $k_{t,1}=k_1|_A=1_A:A\times I\rightarrow X$, by homotopy extension property of $(X\times I,A\times I)$,\marginnote{Since $(X,A)$ has homotopy extension property, $X\times I$ can be deformation retracted to $X\times \{0\}\cup A\times I$, and thus $X\times I\times I$ can be deformation retracted to $X\times I\times \{0\} \cup A\times I\times I$, thus $(X\times I,A\times I)$ has homotopy extension property.} we can extend this to $\tilde{k}_{t,u}:X\times I\times I\rightarrow X$ which satisfies $\tilde{k}_{t,0}=k_t$. Finally define
\begin{equation}
\tilde{h}_t=\begin{cases}
k_{0,3t},&t\in [0,\frac{1}{3}]\\
k_{3t-1,1},&t\in [\frac{1}{3},\frac{2}{3}]\\
k_{1,3-3t},&t\in [\frac{2}{3},1]
\end{cases}
\end{equation}
which is continuous. since $\tilde{h}_0=g_1\circ f$ and $\tilde{h}_1=h_1=1_X$, we have homotopy $g_1\circ f \simeq 1_X \textrm{ rel } A$.

Now, since $f\circ g\simeq f\circ g_1\simeq 1_Y$, we may redo the above argument, which gives a map $f_1$ which is homotopic with $f$, $f_1|_A=1_A$, and $f_1\circ g_1\simeq 1_Y \textrm{ rel } A$. Since $g_1\circ f\simeq 1_X \textrm{ rel }A$, we get $f_1\simeq f_1\circ g_1\circ f \simeq f \textrm{ rel }A$. Therefore $f_1\circ g_1\simeq f\circ g_1\simeq 1_Y \textrm{ rel }A$.
\end{proof}
\noindent\rule{\textwidth}{1pt}
\newline