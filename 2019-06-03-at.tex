\mytitle{Algebraic Topology}
\begin{defn} For an action of a group $G$ on a space $X$, if each $x\in X$ has an open neighborhood $U$ such that all but finitely many images $g(U)$ for $g\in G$ are disjoint for different $g$, then we call this a \textbf{properly discontinuous action}.
\end{defn}
\begin{prop} If a group $G$ acts on a Hausdorff space $X$ freely\marginnote{\begin{defn}Take a group $G$ acting on a set $X$. A point $x\in X$ is a \textbf{fixed point} if $g\cdot x=x$ for some nontrivial $g\in G$. If $X$ has no fixed point, then the action is called a \textbf{free action}.\end{defn}} and properly discontinuously, then the action is a covering space action.
\end{prop}
\begin{proof} Suppose that for some $x\in X$ and its open neighborhood $U$ all but finitely many $g\in G$ satisfies $U\cap g\cdot U=\emptyset$. Then for any $g'\in G$, all but finitely many $g''\in G$ satisfies $g'\cdot U\cap g''\cdot U=\emptyset$ because $g'\cdot U\cap g''\cdot U$ is homeomorphic to $U\cap g'^{-1}g''\cdot U$. Therefore it is enough to show that $U\cap g\cdot U=\emptyset$ for all $g\in G$. We can choose a finite elements $g_1,\cdots,g_n\in G$ such that if $g'\neq g_i$ for all $i=1,\cdots, n$ then $U\cap g'\cdot U=\emptyset$. Now since the action is free, $g_i\cdot x\neq x$ for all $i=1,\cdots,n$, and since $X$ is Hausdorff there is disjoint open neighborhoods $U_i,V_i$ of $x,g_i\cdot x$ respectively. Now take $V=U\cap \bigcap_i U_i \cap \bigcap_i g_i^{-1}\cdot V_i$. This is intersection of finitely many open sets, hence open neighborhood of $x$, and $V\cap g'\cdot V=\emptyset$ for all $g'\neq g_i$. Furthermore, $V\cap g_i\cdot V\subset U_i\cap g_i\cdot(g_i^{-1}\cdot V_i)=U_i\cap V_i=\emptyset$. Therefore the action is a covering space action.
\end{proof}
\begin{defn} Let $G$ be a group which has a representation $\langle g_\alpha|r_\beta\rangle$. The \textbf{Cayley graph of $G$ with respect to the generators $g_\alpha$} is a graph whose vertices are the elements of $g$ and edges join $gg_\alpha$ for each generators $g_\alpha$. For each relations $r_\beta$, which is represented as a loop in Cayley graph, attaching 2-cell gives a cell complex $\tilde{X}_G$, which is called the \textbf{Cayley complex of $G$}.
\end{defn}
\begin{prop} Let $G=\langle g_\alpha|r_\beta\rangle$ is a group. Then the Cayley graph of $G$ with respect to the generators $g_\alpha$ is path connected, and the group $G$ acts on the Cayley complex of $G$, $\tilde{X}_G$, by the multiplication on the left, which is a covering space action. Also $\tilde{X}_G$, is the universal cover of $X_G$, a 2-dimensional cell complex which has fundamental group $G$ which is shown in the previous corollary.
\end{prop}
\begin{proof}
Since every elements can be represented as the finite multiplication of generators, each vertex and vertex $\{e\}$ are path connected, thus the Cayley graph of $G$ is path connected. Now consider the action of $G$ on $\tilde{X}_G$ as, for $g\in G$, $g$ takes a vertex $g'\in G$ to $gg'$, an edge connecting $g',g''$ to an edge connecting $gg',gg''$, and a 2-cell with boundary loop passing $g_1,\cdots,g_n$ to a 2-cell with boundary loop passing $gg_1,\cdots,gg_n$. Now take a point of Cayley graph. If the point is a vertex, then choose an open neighborhood as a point and $1/3$ of its neighboring edges. If the point is not on vertex but on edge, then choose an open neighborhood which is totally contained in edge but does not contains any vertex, which is possible since the edge without its endpoints is an open set. If the point is neither on vertex nor on edge but on 2-cell, then choose an open neighborhood which is totally contained in 2-cell but does not contains any edge or vertex, which is possible since deleting boundary from 2-cell gives an open set. By this procedure, we can find an open neighborhood of every points in Cayley graph whose image of action of all elements of $G$ is disjoint.  Thus this action is a covering space action. Finally, $\tilde{X}_G/G$ and $X_G$ are homeomorphic, since for the map $p:\tilde{X}_G\rightarrow X_G$ taking all the vertices into one point, all the edges connecting $g,g_\alpha$ to the edges representing $g_\alpha$, and all the 2-cells with boundary loop passing $g,gg_1,\cdots,gg_1\cdots g_n$ to the 2-cells with boundary loop passing $g_1,\cdots,g_n$, then this map is quotient map where all the orbits of the action of $G$ on $\tilde{X}_G$ are quotiented. Now due to the previous proposition, $G\simeq \pi_1(\tilde{X}_G/G)/p_*(\pi_1(\tilde{X}_G))\simeq \pi_1(X_G)/p_*(\pi_1(\tilde{X}_G))\simeq G/p_*(\pi_1(\tilde{X}_G))$, thus $p_*(\pi_1(\tilde{X}_G))=0$, and since $p_*$ is injective, $\pi_1(\tilde{X}_G)=0$.
\end{proof}
\noindent\rule{\textwidth}{1pt}
\newline