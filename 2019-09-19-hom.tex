\mytitle{An introduction to homological algebra}
\begin{exer} Let $M$ be an $R$-module and $Z(M)$ be a product of copies $I_0=\Hom_{\mathsf{Ab}}(R,\mathbb{Q}/\mathbb{Z})$, indexed by the set $\Hom_R(M,I_0)-0$. Then $Z(M)$ is injective since it is a product of injectives, and there is a canonical map $e_M:M\rightarrow Z(M)$. Show that $e_M$ is an injection, and thus, show that $R-\mathsf{mod}$ has enough injectives.
\end{exer}
\begin{solution} Take $r\in R$. Notice that $f\in \Hom_{\mathsf{Ab}}(R,\mathbb{Q}/\mathbb{Z})$-th component of $e_M(r)$ is $f(r)$. Define $f:r\mathbb{Z}\rightarrow \mathbb{Q}/\mathbb{Z}$ with $f(r)$ as some nonzero value in $\mathbb{Q}/\mathbb{Z}$, for example, $\left[\frac{1}{2}\right]$. Now since we have a map $r\mathbb{Z}\rightarrow R$ and $\mathbb{Q}/\mathbb{Z}$ is injective, we can extend $f$ to a map $f':R\rightarrow \mathbb{Q}/\mathbb{Z}$, which has $f'(r)\neq 0$. Therefore $e_M$ is injective.
\end{solution}

\begin{defn} A set $I$ is a \textbf{directed set} if there is a relation $\leq$ such that:
\begin{enumerate}
\item $i\leq i$ for all $i\in I$;
\item $i\leq j, j\leq k$ then $i\leq k$ for all $i,j,k\in I$;
\item For any $i,j\in I$, there is $k\in I$ such that $i\leq k$ and $j\leq k$.
\end{enumerate}
For a directed set $I$, let $\{A_i:i\in I\}$ be a family of objects of category $\mathsf{A}$ indexed by $I$. Let $f_{ij}:A_i\rightarrow A_j$ be a morphism for all $i\leq j$ with:
\begin{enumerate}
\item $f_{ii}$ is the identity of $A_i$;
\item $f_{ik}=f_{jk}\circ f_{ij}$ for all $i\leq j\leq k$.
\end{enumerate}
Then the pair $\langle A_i,f_{ij}\rangle$ is called a \textbf{direct system over $I$}.

Let $\langle A_i,f_{ij}\rangle$ be a direct system in $\mathsf{A}$. A \textbf{target} of $\langle A_i,f_{ij}\rangle$ is a pair $\langle A,\phi_i\rangle$, where $A$ is an object in $\mathsf{A}$ and $\phi_i:A_i\rightarrow A$ are morphisms satisfying $\phi_i=\phi_j\circ f_{ij}$ for all $i\leq j$. A \textbf{direct limit} of $\langle A_i,f_{ij}\rangle$ is a target $\langle A,\phi_i\rangle$ such that for all target $\langle B,\psi_i\rangle$, there is a unique morphism $u:X\rightarrow Y$ such that $u\circ \phi_i=\psi_i$ for all $i$.
\begin{equation}
\begin{tikzcd}
A_i\arrow{rr}{f_{ij}} \arrow{rd}{\phi_i} \arrow[rdd, bend right,"\psi_i"]&&A_j\arrow{ld}{\phi_j}\arrow[ldd, bend left,"\psi_j"]\\
&A\arrow[d, dotted,"u"]&\\
&B&
\end{tikzcd}
\end{equation}
We write $A=\lim_{\rightarrow} A_i$.
\end{defn}

\begin{defn} For a topological space $X$ and a sheaf $F$, a \textbf{stalk} of a sheaf $F$ at a point $x\in X$ is the abelian group $F_x:\lim_{\rightarrow}\{F(U):x\in U\}$.
\end{defn}

\begin{defn} For an abelian group $A$ and a topological space $X$, the \textbf{skyscraper sheaf} $x_*A$ at the point $x\in X$ is the presheaf
\begin{equation}
(x_* A)(U)=\begin{cases}
A&x\in U\\
0&x\notin U
\end{cases}
\end{equation}
\end{defn}

\begin{exer} Show that $x_*A$ is a sheaf and that
\begin{equation}
\Hom_{\mathsf{Ab}}(F_x,A)\simeq \Hom_{\textrm{Sheaves(X)}}(F,x_* A)
\end{equation}
for every sheaf $F$. Thus if $A_x$ is an injective abelian group, then $x_* A_x$ is an injective object in $\textrm{Sheaves}(X)$ for each $x$, and that $\prod_{x\in X} x_* A_x$ is also injective.
\end{exer}
\begin{solution} Restricting $x_*A(U)$ to $U_i$ gives $0$ if $x\notin U_i$ and $A$ if $x\in U_i$, which shows that $x_*A$ is a sheaf. To show the isomorphism, take $\tau:\Hom_{\mathsf{Ab}}(F_x,A)\rightarrow \Hom_{\textrm{Sheaves(X)}}(F,x_* A)$ as $\tau(f)(U)$ is a zero map if $x\notin U$ and $\tau(f)(U)$ is a composition map $F(U)\rightarrow F_x\xrightarrow{f} A$. Now take $\mu:\Hom_{\textrm{Sheaves(X)}}(F,x_* A)\rightarrow \Hom_{\mathsf{Ab}}(F_x,A)$ as the map which is uniquely generated by the direct limit. Since each construction gives the same commuting diagram, $\tau\circ \mu$ and $\mu\circ \tau$ are identity maps. Thus the stalk functor and skyscraper sheaf functor are left and right adjoints respectively. Since the stalk functor is exact,\marginnote{Stalk functor is exact since the direct limit functor is exact.} the previous proposition shows the last statement.
\end{solution}

\begin{exmp} $\textrm{Sheaves}(X)$ has enough injectives. Indeed, given a fixed sheaf $F$, choose an injection $F_x\rightarrow I_x$ with $I_x$ injective in $\mathsf{Ab}$ for each $x\in X$. Combining the natural maps $F\rightarrow x_* F_x$ with $x_* F_x\rightarrow x_* I_x$ gives a map from $F$ to the injective sheaf $I=\prod_{x\in X}x_* I_x$. The map $F\rightarrow I$ can be shown that it is an injection.
\end{exmp}


\noindent\rule{\textwidth}{1pt}
\newline
